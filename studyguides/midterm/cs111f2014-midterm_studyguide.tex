%!TEX root=cs111F2014-lab09.tex
% mainfile: cs111F2014-lab09.tex

% CS 111 style
% Typical usage (all UPPERCASE items are optional):
%       \input 111pre
%       \begin{document}
%       \MYTITLE{Title of document, e.g., Lab 1\\Due ...}
%       \MYHEADERS{short title}{other running head, e.g., due date}
%       \PURPOSE{Description of purpose}
%       \SUMMARY{Very short overview of assignment}
%       \DETAILS{Detailed description}
%         \SUBHEAD{if needed} ...
%         \SUBHEAD{if needed} ...
%          ...
%       \HANDIN{What to hand in and how}
%       \begin{checklist}
%       \item ...
%       \end{checklist}
% There is no need to include a "\documentstyle."
% However, there should be an "\end{document}."
%
%===========================================================
\documentclass[11pt,twoside,titlepage]{article}
%%NEED TO ADD epsf!!
\usepackage{threeparttop}
\usepackage{graphicx}
\usepackage{latexsym}
\usepackage{color}
\usepackage{listings}
\usepackage{fancyvrb}
%\usepackage{pgf,pgfarrows,pgfnodes,pgfautomata,pgfheaps,pgfshade}
\usepackage{tikz}
\usepackage[normalem]{ulem}
\tikzset{
    %Define standard arrow tip
%    >=stealth',
    %Define style for boxes
    oval/.style={
           rectangle,
           rounded corners,
           draw=black, very thick,
           text width=6.5em,
           minimum height=2em,
           text centered},
    % Define arrow style
    arr/.style={
           ->,
           thick,
           shorten <=2pt,
           shorten >=2pt,}
}
\usepackage[noend]{algorithmic}
\usepackage[noend]{algorithm}
\newcommand{\bfor}{{\bf for\ }}
\newcommand{\bthen}{{\bf then\ }}
\newcommand{\bwhile}{{\bf while\ }}
\newcommand{\btrue}{{\bf true\ }}
\newcommand{\bfalse}{{\bf false\ }}
\newcommand{\bto}{{\bf to\ }}
\newcommand{\bdo}{{\bf do\ }}
\newcommand{\bif}{{\bf if\ }}
\newcommand{\belse}{{\bf else\ }}
\newcommand{\band}{{\bf and\ }}
\newcommand{\breturn}{{\bf return\ }}
\newcommand{\mod}{{\rm mod}}
\renewcommand{\algorithmiccomment}[1]{$\rhd$ #1}
\newenvironment{checklist}{\par\noindent\hspace{-.25in}{\bf Checklist:}\renewcommand{\labelitemi}{$\Box$}%
\begin{itemize}}{\end{itemize}}
\pagestyle{threepartheadings}
\usepackage{url}
\usepackage{wrapfig}
\usepackage{hyperref}
%=========================
% One-inch margins everywhere
%=========================
\setlength{\topmargin}{0in}
\setlength{\textheight}{8.5in}
\setlength{\oddsidemargin}{0in}
\setlength{\evensidemargin}{0in}
\setlength{\textwidth}{6.5in}
%===============================
%===============================
% Macro for document title:
%===============================
\newcommand{\MYTITLE}[1]%
   {\begin{center}
     \begin{center}
     \bf
     CMPSC 111\\Introduction to Computer Science I\\
     Fall 2014\\
     %Janyl Jumadinova\\
     %\url{http://cs.allegheny.edu/~jjumadinova/111}
     \medskip
     \end{center}
     \bf
     #1
     \end{center}
}
%================================
% Macro for headings:
%================================
\newcommand{\MYHEADERS}[2]%
   {\lhead{#1}
    \rhead{#2}
    \immediate\write16{}
    \immediate\write16{DATE OF HANDOUT?}
    \read16 to \dateofhandout
    \lfoot{\sc Handed out on \dateofhandout}
    \immediate\write16{}
    \immediate\write16{HANDOUT NUMBER?}
    \read16 to\handoutnum
    \rfoot{Handout \handoutnum}
   }

%================================
% Macro for bold italic:
%================================
\newcommand{\bit}[1]{{\textit{\textbf{#1}}}}

%=========================
% Non-zero paragraph skips.
%=========================
\setlength{\parskip}{1ex}

%=========================
% Create various environments:
%=========================
\newcommand{\PURPOSE}{\par\noindent\hspace{-.25in}{\bf Purpose:\ }}
\newcommand{\SUMMARY}{\par\noindent\hspace{-.25in}{\bf Summary:\ }}
\newcommand{\DETAILS}{\par\noindent\hspace{-.25in}{\bf Details:\ }}
\newcommand{\HANDIN}{\par\noindent\hspace{-.25in}{\bf Hand in:\ }}
\newcommand{\SUBHEAD}[1]{\bigskip\par\noindent\hspace{-.1in}{\sc #1}\\}
%\newenvironment{CHECKLIST}{\begin{itemize}}{\end{itemize}}

\begin{document}
\MYTITLE{Information about the Midterm Exam\\Exam Date: Thursday, October 16, 2014 at 2:30 pm}
%\MYHEADERS{First Exam Information}{}

%\subsection*{Honor Code Reminder---See Last Page}
\subsection*{Overview}

Since the exam will be during the laboratory session, we will not have a
lab on October 16. The exam is closed book and closed notes. The exam will
cover the following material:

\begin{itemize}
\item
Chapter 1 in Lewis \& Loftus. 
\item
Chapter 2 in Lewis \& Loftus, Sections 2.1--2.9. 
\item 
Chapter 3 in Lewis \& Loftus, Sections 3.1--3.8.
Know the terms in these sections. Be familiar with the methods of the {\tt String}
class, particularly {\tt length}, {\tt substring} (both forms), 
{\tt charAt}, {\tt equals}, {\tt toUpperCase}, {\tt toLowerCase}, and {\tt replace}
(see lab 5---the ``DNA'' lab).
Know the methods of the {\tt Random} class: {\tt nextInt}, {\tt nextFloat}, and {\tt nextDouble}.
\item
Definition, basic commands in the Linux
operating system; editing in {\tt gvim}, compiling and executing programs
in Linux; basic commands using Bitbucket.
\item
Your class notes and lecture slides, labs 1--5, practicals 1--3, handouts from class, and quiz 1.
\end{itemize}

%\subsection*{Format}
\noindent The exam will be a mix of questions such as fill in the blank, short answer, true/false, and completion.  The
emphasis will be on the following topics:

\begin{itemize}
\item
Fundamental concepts (e.g., definitions and background)
\item
Basic techniques (e.g., editing, compiling, and running programs; using
  files and directories; using Bitbucket with the command-line {\tt git} programs)
\item
Understanding Java programs (given something written in Java, understand
what it does and be able to precisely describe its output).
\item
Composing Java statements and programs, given a description of what
should be done.
\end{itemize}

\noindent
Students are required to fully adhere to the Honor Code during the completion of the midterm examination. More details
about the Allegheny College Honor Code are provided on the next page.

\newpage

\begin{center}
\bf
The Honor Code as Applied to Examinations\\
(See
\url{http://sites.allegheny.edu/deanofstudents/student-conduct-system}\\
for more information)
\end{center}
{\small
\subsection*{Article III}

\noindent{\bf Section 1}

If one student observes another committing what appears to be an act of
dishonesty in academic work it is the observer's responsibility to take
the appropriate action. Students are encouraged to inform either the
instructor or a member of the Honor Committee. However, whatever action
the observer takes must fulfill the obligation to uphold the integrity
of the College community. Failure to do so is as injurious to the honor
of the College community as is the observed act of dishonesty and
constitutes an infraction of the Honor Code.

\noindent{\bf Section 2}

The following practices are considered to be violations of the Honor
Code in examinations, tests, quizzes, and in laboratory and computing
exercises: any attempt to receive or give unauthorized assistance from
written, printed, or recorded aids, from any person, or from another's
work.
\ldots
\subsection*{Article IV}

\noindent{\bf Section 1}

Tests and examinations at Allegheny need not be proctored. Instructors
may remain in the room or in a nearby room but must remain in the
building to be available to answer questions that may arise during the
course of the examination.

\noindent{\bf Section 2}

Examinations are confined to the building in which they are given.
Students shall have freedom of movement within that building. Students
may not leave the building or take materials related to the exams into
restrooms unless explicitly permitted to do so by the instructor, or
unless the instructor declares the test to be written at home or other
parts of the campus. Additionally, exams may not be taken behind a
locked door. It is the student’s responsibility to ensure that the door
to the room remains unlocked during the entire exam.

\noindent{\bf Section 3}

Regardless of where the test or examination is taken, the student is
responsible for obtaining any changes or corrections. Instructors are
not under obligation to search out students to provide this information.
Furthermore, the exam must be handed in at the time requested.

\noindent{\bf Section 4}

In recognition of the responsibilities of the Honor Program, a student,
when submitting a test or paper, shall sign both the pledge and full
name in signature. If a student neglects to do this, the instructor must
notify the student and allow an opportunity for signing the paper.
Moreover, work is not to be considered as graded until the pledge and
signature appear.
}

\end{document}
