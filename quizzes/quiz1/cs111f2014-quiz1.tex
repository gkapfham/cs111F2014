%!TEX root=cs111f2014-quiz1.tex
% mainfile: cs111f2014-quiz1.tex

%!TEX root=cs111F2014-lab09.tex
% mainfile: cs111F2014-lab09.tex

% CS 111 style
% Typical usage (all UPPERCASE items are optional):
%       \input 111pre
%       \begin{document}
%       \MYTITLE{Title of document, e.g., Lab 1\\Due ...}
%       \MYHEADERS{short title}{other running head, e.g., due date}
%       \PURPOSE{Description of purpose}
%       \SUMMARY{Very short overview of assignment}
%       \DETAILS{Detailed description}
%         \SUBHEAD{if needed} ...
%         \SUBHEAD{if needed} ...
%          ...
%       \HANDIN{What to hand in and how}
%       \begin{checklist}
%       \item ...
%       \end{checklist}
% There is no need to include a "\documentstyle."
% However, there should be an "\end{document}."
%
%===========================================================
\documentclass[11pt,twoside,titlepage]{article}
%%NEED TO ADD epsf!!
\usepackage{threeparttop}
\usepackage{graphicx}
\usepackage{latexsym}
\usepackage{color}
\usepackage{listings}
\usepackage{fancyvrb}
%\usepackage{pgf,pgfarrows,pgfnodes,pgfautomata,pgfheaps,pgfshade}
\usepackage{tikz}
\usepackage[normalem]{ulem}
\tikzset{
    %Define standard arrow tip
%    >=stealth',
    %Define style for boxes
    oval/.style={
           rectangle,
           rounded corners,
           draw=black, very thick,
           text width=6.5em,
           minimum height=2em,
           text centered},
    % Define arrow style
    arr/.style={
           ->,
           thick,
           shorten <=2pt,
           shorten >=2pt,}
}
\usepackage[noend]{algorithmic}
\usepackage[noend]{algorithm}
\newcommand{\bfor}{{\bf for\ }}
\newcommand{\bthen}{{\bf then\ }}
\newcommand{\bwhile}{{\bf while\ }}
\newcommand{\btrue}{{\bf true\ }}
\newcommand{\bfalse}{{\bf false\ }}
\newcommand{\bto}{{\bf to\ }}
\newcommand{\bdo}{{\bf do\ }}
\newcommand{\bif}{{\bf if\ }}
\newcommand{\belse}{{\bf else\ }}
\newcommand{\band}{{\bf and\ }}
\newcommand{\breturn}{{\bf return\ }}
\newcommand{\mod}{{\rm mod}}
\renewcommand{\algorithmiccomment}[1]{$\rhd$ #1}
\newenvironment{checklist}{\par\noindent\hspace{-.25in}{\bf Checklist:}\renewcommand{\labelitemi}{$\Box$}%
\begin{itemize}}{\end{itemize}}
\pagestyle{threepartheadings}
\usepackage{url}
\usepackage{wrapfig}
\usepackage{hyperref}
%=========================
% One-inch margins everywhere
%=========================
\setlength{\topmargin}{0in}
\setlength{\textheight}{8.5in}
\setlength{\oddsidemargin}{0in}
\setlength{\evensidemargin}{0in}
\setlength{\textwidth}{6.5in}
%===============================
%===============================
% Macro for document title:
%===============================
\newcommand{\MYTITLE}[1]%
   {\begin{center}
     \begin{center}
     \bf
     CMPSC 111\\Introduction to Computer Science I\\
     Fall 2014\\
     %Janyl Jumadinova\\
     %\url{http://cs.allegheny.edu/~jjumadinova/111}
     \medskip
     \end{center}
     \bf
     #1
     \end{center}
}
%================================
% Macro for headings:
%================================
\newcommand{\MYHEADERS}[2]%
   {\lhead{#1}
    \rhead{#2}
    \immediate\write16{}
    \immediate\write16{DATE OF HANDOUT?}
    \read16 to \dateofhandout
    \lfoot{\sc Handed out on \dateofhandout}
    \immediate\write16{}
    \immediate\write16{HANDOUT NUMBER?}
    \read16 to\handoutnum
    \rfoot{Handout \handoutnum}
   }

%================================
% Macro for bold italic:
%================================
\newcommand{\bit}[1]{{\textit{\textbf{#1}}}}

%=========================
% Non-zero paragraph skips.
%=========================
\setlength{\parskip}{1ex}

%=========================
% Create various environments:
%=========================
\newcommand{\PURPOSE}{\par\noindent\hspace{-.25in}{\bf Purpose:\ }}
\newcommand{\SUMMARY}{\par\noindent\hspace{-.25in}{\bf Summary:\ }}
\newcommand{\DETAILS}{\par\noindent\hspace{-.25in}{\bf Details:\ }}
\newcommand{\HANDIN}{\par\noindent\hspace{-.25in}{\bf Hand in:\ }}
\newcommand{\SUBHEAD}[1]{\bigskip\par\noindent\hspace{-.1in}{\sc #1}\\}
%\newenvironment{CHECKLIST}{\begin{itemize}}{\end{itemize}}

\begin{document}
\thispagestyle{empty}
\MYTITLE{Quiz\\
25-26 September 2014\\
100 points}

\begin{center}
\parbox{4.5in}{
Name (printed): \underline{\hspace{2.5in}}

\bigskip
Signature for Honor Code: \underline{\hspace{2.5in}}
}

\end{center}

\begin{quote}
The quiz is closed book, closed notes, and closed computer. 

Place all answers on the quiz pages.
\end{quote}

\begin{enumerate}
\item {\bf [9 points]}
\begin{enumerate}
\item {\bf [3 points]} 
Name and explain three types of computer hardware.
\bigskip
\bigskip
\bigskip
\bigskip
\item {\bf [3 points]} 
Explain the fetch-decode-execute cycle that the computer performs.

\bigskip
\bigskip
\bigskip
\bigskip
\item {\bf [3 points]} 
Give three examples of a Java keyword, explaining the meaning of each.
\bigskip
\bigskip
\bigskip
\bigskip
\end{enumerate}
\item {\bf [9 points]}
\begin{enumerate}
\item {\bf [3 points]} 
  What are the input(s), output(s), and behavior(s) of the Java compiler?
\bigskip
\bigskip
\bigskip
\bigskip
\item {\bf [3 points]} 
How many distinct values can be represented using 5 bits? Why?
\bigskip
\bigskip
\bigskip
\bigskip
\item {\bf [3 points]} 
What is a primitive data type? Give two examples, explaining their purpose.
\end{enumerate}
\bigskip
\bigskip
\bigskip
\bigskip

\item {\bf [6 points]}
Write the Linux commands to create a new directory directory named ``{\tt quiz}'',
change to that directory, and begin editing a file named ``{\tt Quiz.java}'' using
the GVim editor.

\vspace{2in}

\item {\bf[10 points]} Using correct operator precedence for the Java programming language, what values would be printed
  by the following two {\tt print} statements?

\begin{verbatim}
System.out.print ( ( 10 + 4 ) * 3 % 8 * ( 2 + 3 ) ); 		
\end{verbatim}

\vspace*{-.175in}
The output is: \mbox{\underline{\hspace{3in}}}

\begin{verbatim}
System.out.print ( 10 + 4 * 3 % 8 * 2 + 3);				    
\end{verbatim}	

\vspace*{-.175in}
The output is: \mbox{\underline{\hspace{3in}}}
\item {\bf [15 points]}

What output is produced by the following Java program? Show everything that
is printed by the program's ``{\tt System.out.println}'' statements.
\begin{verbatim}
          public class Quiz
          {
              public static void main(String[] args)
              {
                  double x = 2.0;
                  int y = 10;
          
                  x = x * y;
                  y = y % 7;
          
                  System.out.println("x = " + x);
                  System.out.println("y = " + y);
                  System.out.println("\\ \\ \n / /");
              }
          }
\end{verbatim}
\vspace*{-.05in}
The output is:
\vspace{1in}

\item {\bf [25 points]}
Write the Java statements (NOT a complete program!) needed to:
\begin{itemize}
\item
Declare and initialize two {\tt int} variables named {\tt a} and {\tt b}
(initialize them to any values you wish, as long as they
are legal {\tt int} values)
\item
Declare a {\tt double} value named {\tt avg} and set it equal to the
average of {\tt a} and {\tt b}. For instance, if you set {\tt a} to 1 and
{\tt b} to 2, your Java program should compute the value of {\tt avg} as
1.5. (You must use an arithmetic expression for the 
average of {\tt a} and {\tt b}; don't
just ``compute the answer in your head'' and assign a constant value to {\tt avg}.)
\item
Print the phrase ``\verb$avg = $'', followed by the value of the variable 
{\tt avg}.
\end{itemize}

\vspace{3.5in}

\item {\bf [12 points]}
Given the declarations:\\

{\tt int value1;}\\
{\tt int num1 = 17, num2 = 5;}\\
{\tt double value2;}\\
{\tt double num3 = 12.0, num4 = 2.34;}\\ 

 what result is stored by each of the following assignment statements? Please explain why.
 \begin{itemize}
 \item {\tt value1 = num1 / num2;}
 
\vspace{0.3in}
 \item {\tt value2 = num3 / num2;}
 
\vspace{0.3in}
 \item {\tt value1 = (int) num3 / num2;}
 
\vspace{0.3in}
 \end{itemize}


\item {\bf [3 points]}
Assume you have written a program named ``{\tt Quiz.java}'' and that you have 
successfully run the ``{\tt javac}'' command without receiving any  errors. 
Which of the following commands will run (i.e., execute) your program?
\begin{enumerate}
\item \verb$Quiz.java$

\medskip
\item \verb$java Quiz.java$

\medskip
\item \verb$javac Quiz.class$

\medskip 
\item \verb$gvim Quiz.java$

\medskip
\item None of the above; the correct answer is \underline{\hspace{3in}}
\end{enumerate}


\bigskip
\bigskip


\item {\bf [3 points]}
  The .java extension on a file means that the file:
  \begin{enumerate}
    \item contains Java source code
      \medskip 
    \item contains Java byte code
      \medskip
    \item contains HTML
      \medskip 
    \item is produced by the Java compiler called ``{\tt javac}''
      \medskip
    \item None of the above; the correct answer is \underline{\hspace{3in}}
  \end{enumerate}

\bigskip
\bigskip

\item {\bf [3 points]}
  Which of the following is needed to allow a Java program to use the {\tt Date} class?
  \begin{enumerate}
    \item {\tt import Date;}
      \medskip 
    \item {\tt import java.calendar.Date;}
      \medskip 
    \item {\tt insert java.Date;}
      \medskip
    \item {\tt import java.util.Date;}
      \medskip
    \item None of the above; the correct answer is \underline{\hspace{3in}}
  \end{enumerate}

\bigskip
\bigskip

% \item {\bf [3 points]}
% Which of the following is an appropriate \textbf{main} class definition ?
% \begin{enumerate}
% \item {\tt{public class main ( String [ ] args  )}}
% \medskip 
% \item {\tt{public static void main( String [ ] args )}}
% \medskip 
% \item {\tt{public static class main ( String [ ] args )}}
% \medskip
% \item {\tt{public static void main ( )}}
% \medskip
% \item None of the above; the correct answer is \underline{\hspace{3in}}
% \end{enumerate}

\item {\bf [3 points]}
  Which of the following expressions will store the value of 21 in {\tt result}?
\begin{enumerate}
  \item {\tt result = 3 * ((19 - 4) / 2);}
\medskip 
  \item {\tt result = 3 * ((18 - 4) / 2);}
\medskip 
  \item {\tt result = 3 * ((19 - 8) / 2);}
\medskip
  \item {\tt result = 4 * ((18 - 4) / 2);}
\medskip
\item None of the above; the correct answer is \underline{\hspace{3in}}
\end{enumerate}


\bigskip
\bigskip

\end{enumerate}

\end{document}
