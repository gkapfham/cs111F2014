\input{111pre}
\begin{document}
\thispagestyle{empty}
\MYTITLE{Quiz 1 - Review Sheet}
%\MYHEADERS{Name: \underline{\hspace{2.5in}}}{17 Sep.}

\begin{quote}
\textcolor{red}{The quiz will be during your practical session, either on Thursday (September 25) at 1:30 pm or on Friday (September 26) at 9 am. }

\textbf{We will still have our regular lab on Thursday at 2:30 pm.}

The quiz will be closed book, closed notes, closed computer. 

\end{quote}

\textbf{Content Covered} \\
This quiz covers material primarily found in chapters 1.1-1.4, 2.1-2.6. \\

\textbf{Quiz Format}\\
The quiz will consist of a mixture of the following types of questions:
\begin{itemize}
\item Multiple choice
\item Short answer (provide 2-3 sentence explanation or definition of terms/concepts)
\item Short Programming (provide valid Java statements given a brief problem description)
\item Program Analysis (given a segment of code, determine what would  the output be)
\end{itemize}

\textbf{Material}
\begin{itemize}
\item Basic UNIX commands for creating, changing directories, creating files, etc.
\item File naming conventions for Java programs
\item The differences between machine, assembly, and high level programming languages
\item The usage of $javac$ and $java$
\item Basic output using $System.out.print()$,  $System.out.println()$
\item Basic input using the $Scanner$ class, how to create a new $Scanner$ object
\item Basic program structure ($import$ statements, structure of $main$ method, etc.)
\item Arithmetic operators
\item Primitive data types
\item Declaration, assignment statements
\item Expressions
\item Escape sequences
\item Comment format 
\end{itemize}

\noindent \textbf{Some sample questions:} 
(These do not represent a complete sample quiz.)

\begin{enumerate}
\item How many unique values can be represented using four binary digits (bits)?

\item In Java, what is an {\em identifier}?
\item In Java, what is an {\em escape sequence}? Give an example.

\item If you have a Java program named ``{\tt JavaProg.java}'', what command must you
type to compile it?

\item
If you are in your home directory and you type the commands ``{\tt cd mylabs}'',
``{\tt cd lab3}'', and ``{\tt cd ..}'', what directory do you end up in? (Assume
that all of the directories exist.)

\item
What is ``INSERT'' mode and how do you get into it and out of it?

\item 
What output is produced by the following Java program? Show everything that
is printed by the program's ``{\tt System.out.println}'' statements.
\begin{verbatim}
          public class Quiz
          {
              public static void main(String[] args)
              {
                  int i = 10;
                  double x = .5;
          
                  x = x + 1;
                  i = i / 3;
          
                  System.out.println("i = " + i);
                  System.out.println("x = " + x);
              }
          }
\end{verbatim}

\item 
Write the Java statements (NOT a complete program!) needed to 
\begin{itemize}
\item
declare three variables named {\tt a}, {\tt b}, and 
{\tt c} of any primitive data type; assign them any values you wish (as long 
as the values are legal for the data type you selected)
\item
Print the variables (do not use literals---use the variable names), all on one
line, each surrounded by
double-quote marks, with a space separating the values. For example, 
if your three variables contain the real numbers 3.2, 5.09, and -6.2, your
program should print:
\begin{center}
\verb$"3.2" "5.09" "-6.2"$
\end{center}
\end{itemize}

\end{enumerate}

\end{document}