%!TEX root=cs111f2014-quiz2.tex
% mainfile: cs111f2014-quiz2.tex

\input{111pre}
\begin{document}
\thispagestyle{empty}
\MYTITLE{Quiz\\
13--14 November 2014\\
100 points}

\begin{center}
\parbox{4.5in}{
Name (printed): \underline{\hspace{2.5in}}

\bigskip
Signature for Honor Code: \underline{\hspace{2.5in}}
}

\end{center}

\begin{quote}
The quiz is closed book, closed notes, and closed computer. 

Place all answers on the quiz pages.
\end{quote}

\begin{enumerate}
\item {\bf [9 points]}
\begin{enumerate}
\item {\bf [3 points]} 
Explain the meaning of the term ``infinite loop''; please provide an example.
\bigskip
\bigskip
\bigskip
\item {\bf [3 points]} 
Explain three attributes that could be associated with a {\tt Student} class.
\bigskip
\bigskip
\bigskip
% \bigskip
\item {\bf [3 points]} 
  Explain the meaning of the logical operator associated with the ``{\tt !}'' symbol.
\bigskip
\bigskip
\bigskip
\bigskip
\end{enumerate}
\item {\bf [9 points]}
\begin{enumerate}
\item {\bf [3 points]} 
  What are the input(s), output(s), and behavior(s) of the Java compiler?
\bigskip
\bigskip
\bigskip
\bigskip
\item {\bf [3 points]} 
Please describe three equality and/or relational operators provided by Java.
\bigskip
\bigskip
\bigskip
\bigskip
\item {\bf [3 points]} 
What is a boolean expression? Give two examples, explaining their purpose.
\end{enumerate}
\bigskip
\bigskip
\bigskip
\bigskip

\item {\bf [8 points]}
  Assuming that {\tt a} and {\tt b} are boolean values that can take on the value of either {\tt true} or {\tt false},
  please provide the complete truth tables for ``{\tt a \&\& b}'' and ``{\tt a || b}''.

\vspace{1.2in}

\item {\bf[10 points]} Given the following declarations, what is the value of each of these expressions?

  \begin{verbatim}
  int value1 = 5, value2 = 10;
  boolean done = true;
  \end{verbatim}

  \vspace*{-.35in}
  
\begin{verbatim}
((value1 < value2) || done)
\end{verbatim}

\vspace*{-.175in}
The value is: \mbox{\underline{\hspace{3in}}}

\begin{verbatim}
(done || !done)
\end{verbatim}	

\vspace*{-.175in}
The value is: \mbox{\underline{\hspace{3in}}}

\item {\bf [16 points]}

What output is produced by the following Java program that uses conditional logic? 
\begin{verbatim}
          public class ConditionalLogic {
              public static void main(String[] args) {
                  boolean label = true;
                  double x = 2.0;
                  int y = 10;
                  x = x * y;
                  y = y % 7;

                  if(x<2) {
                    System.out.println("x = " + x);
                    System.out.println("y = " + y);
                  }

                  if(label) {
                    System.out.println("\\ \\ \n / /");
                  }
              }
          }
\end{verbatim}
\vspace*{-.05in}
The output is:
\vspace{1in}

\item {\bf [25 points]}
Write a complete Java program that provides the following features:
\begin{itemize}

  \item Read in four {\tt int} values from the user.  The first three values should be used as inputs to an averaging
    computation and the fourth value will be a ``{\tt cutoff}'' value.

  \item Read in one {\tt String} value, called the ``{\tt operation}'', from the user.

  \item Compute the average of the first three {\tt int} values.

  \item If {\tt operation} contains the value ``greater'', then output the average only if it is greater than the
    user-specified {\tt cutoff} value.

  \item If {\tt operation} contains the string ``smaller'', then output the average only if is less than or equal to the
    user-specified {\tt cutoff} value.

\end{itemize}

\vspace{2.5in}

\item {\bf [15 points]}
  Assume that the following variable declarations appear in the {\tt main} method.\\ 

{\tt int value1 = 10, num1 = 17, num2 = 5;}\\
{\tt double value2 = 19.67;}\\
{\tt double num3 = 12.0, num4 = 2.34;}\\ 

\noindent
How many times will the body of these stand-alone {\tt while} loops execute? Please explain why.
 
 \begin{itemize}
 \item
     \begin{verbatim}
        while(num2 > 0) {
          num2--;
        }
     \end{verbatim}
 
% \vspace{0.3in}
 \item 
 \begin{verbatim}
        while((num1-num2) > 0) {
          value1--;
        }
     \end{verbatim}

% \vspace{0.3in}
 \item 
 \begin{verbatim}
        while(num3 > num4) {
          num2--;
        }
     \end{verbatim}

\vspace{0.3in}
 \end{itemize}


\item {\bf [3 points]}
What natural language description best explains the circumstance in which the ``{\tt else}'' block will
execute when a block of ``{\tt if/else if/else}'' conditional logic executes?
\begin{enumerate}
  \item The {\tt else} block will always execute.

\medskip
\item The {\tt else} block never executes.

\medskip
\item The {\tt else} block executes only when the {\tt if} condition is false.

\medskip 
\item The {\tt else} block executes when one of the {\tt else if} conditions is false.

\medskip 
\item The {\tt else} block executes when at least one of the {\tt else if} conditions is false.

\medskip
\item None of the above; the correct answer is \underline{\hspace{3in}}
\end{enumerate}


\bigskip
\bigskip
\bigskip
\bigskip

\item {\bf [3 points]}
  Suppose that a program running in your terminal seems to be caught in an infinite loop. What command can you
  type to stop this program from continuing to execute?
  \begin{enumerate}
    \item {\tt CTRL-Q}
      \medskip 
    \item {\tt gg=G}
      \medskip
    \item {\tt CTRL-c}
      \medskip 
    \item {\tt CTRL-d}
      \medskip
    \item {\tt terminate}
      \medskip
\item None of the above; the correct answer is \underline{\hspace{3in}}
  \end{enumerate}

\bigskip
\bigskip
\bigskip
% \bigskip

\item {\bf [3 points]}
  Which of the following is needed to allow a Java program to use the {\tt ArrayList}?
  \begin{enumerate}
    \item {\tt import ArrayList;}
      \medskip 
    \item {\tt import java.list.ArrayList;}
      \medskip 
    \item {\tt insert java.ArrayList;}
      \medskip
    \item {\tt import java.util.ArrayList;}
      \medskip
    \item {\tt load ArrayList;}
      \medskip
    \item None of the above; the correct answer is \underline{\hspace{3in}}
  \end{enumerate}

\bigskip
\bigskip
\bigskip

% \item {\bf [3 points]}
% Which of the following is an appropriate \textbf{main} class definition ?
% \begin{enumerate}
% \item {\tt{public class main ( String [ ] args  )}}
% \medskip 
% \item {\tt{public static void main( String [ ] args )}}
% \medskip 
% \item {\tt{public static class main ( String [ ] args )}}
% \medskip
% \item {\tt{public static void main ( )}}
% \medskip
% \item None of the above; the correct answer is \underline{\hspace{3in}}
% \end{enumerate}

% \item {\bf [3 points]}
%   Which of the following expressions will store the value of 21 in {\tt result}?
% \begin{enumerate}
%   \item {\tt result = 3 * ((19 - 4) / 2);}
% \medskip 
%   \item {\tt result = 3 * ((18 - 4) / 2);}
% \medskip 
%   \item {\tt result = 3 * ((19 - 8) / 2);}
% \medskip
%   \item {\tt result = 4 * ((18 - 4) / 2);}
% \medskip
% \item None of the above; the correct answer is \underline{\hspace{3in}}
% \end{enumerate}

\bigskip
\bigskip

\end{enumerate}

\end{document}
