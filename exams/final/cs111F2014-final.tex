\documentclass[12pt]{article}             
\textwidth = 6.5in
\textheight = 9.05in
\topmargin -0.5in
\oddsidemargin 0.0in
\evensidemargin 0.0in

% set it so that subsubsections have numbers and they
% are displayed in the TOC (maybe hard to read, might want to disable)

\usepackage{epsfig}

\usepackage{listings}

\setcounter{secnumdepth}{3}
\setcounter{tocdepth}{3}

% define widow protection 
        
\def\widow#1{\vskip #1\vbadness10000\penalty-200\vskip-#1}

% define a little section heading that doesn't go with any number

\def\littlesection#1{
\widow{2cm}
\vskip 0.5cm
\noindent{\bf #1}
\vskip 0.1cm
\noindent
}

% A paraphrase mode that makes it easy to see the stuff that shouldn't
% stay in for the final proposal

\newdimen\tmpdim
\long\def\paraphrase#1{{\parskip=0pt\hfil\break
\tmpdim=\hsize\advance\tmpdim by -15pt\noindent%
\hbox to \hsize
{\vrule\hskip 3pt\vrule\hfil\hbox to \tmpdim{\vbox{\hsize=\tmpdim
\def\par{\leavevmode\endgraf}
\obeyspaces \obeylines 
\let\par=\endgraf
\bf #1}}}}}

\renewcommand{\baselinestretch}{1.2}    % must go before the begin of doc

\pagestyle{empty}

% go with the way that CC sets the margins

\usepackage{color}

\definecolor{javared}{rgb}{0.6,0,0} % for strings
\definecolor{javagreen}{rgb}{0.25,0.5,0.35} % comments
\definecolor{javapurple}{rgb}{0.5,0,0.35} % keywords
\definecolor{javadocblue}{rgb}{0.25,0.35,0.75} % javadoc

\begin{document}

\lstset{language=Java,
basicstyle=\ttfamily,
keywordstyle=\color{javapurple}\bfseries,
stringstyle=\color{javared},
commentstyle=\color{javagreen},
morecomment=[s][\color{javadocblue}]{/**}{*/},
%numbers=left,
numberstyle=\scriptsize\color{black},
stepnumber=1,
numbersep=7pt,
tabsize=4,
showspaces=false,
showstringspaces=false}

% handle widows appropriately
\def\widow#1{\vskip #1\vbadness10000\penalty-200\vskip-#1}

\begin{center}

Computer Science 111: Introduction to Computer Science I \\
Final Examination \\
Fall 2014 \\

\vspace{.25in}

Name \line(1,0){250}

Pledge \line(1,0){250}

\end{center}

\noindent

% {\bf General Instructions}. Read each question carefully before answering.  The exam is closed book.  Place the answers
% to the questions in Parts I and III in order and in the exam blue books.  Place the answers to the questions in Part II
% in the spaces provided on these exam sheets.  Be careful to allow sufficient time to think through your answers for the
% questions in Parts II and III.  Review your work. Write clearly in pencil.  Erase completely.  You must turn in both
% this exam question sheet and the blue book, with your name printed \mbox{and signed.}

\medskip \noindent \begin{center} The quiz is closed book, closed notes, and closed computer. \\ Please place all
answers on the test sheets.  \end{center}

\newpage

\noindent
{\bf Part I (Short Answers)}

\begin{enumerate}

  \item ({\bf 10 Points}) Furnish good definitions for each of the following:

\begin{enumerate}

\item Algorithm
  \vspace*{1.2in}

\item Escape Sequence 
  \vspace*{1.2in}

\item Object
  \vspace*{1.2in}

\item RGB Value
  \vspace*{1.2in}

\item String Concatenation
  \vspace*{1.2in}

\end{enumerate}

% \item If a picture is made up of 256 possible colors, then how many
%   bits would be needed to store each pixel of the picture?  Your
%   response to this question should clearly explain why you gave the
%   response that you did.

% \item Lewis and Loftus give five steps for solving a problem.  What are they?

\newpage

\item ({\bf 6 Points}) What is the difference between a Java language statement, a Java byte code instruction, and a
  machine language instruction?  Which do you create, which does the Java compiler create, and which does the Java
  virtual machine create?
  \vspace*{4in}

\item ({\bf 4 Points}) Please distinguish between a primitive and a reference data type.  In addition to giving an
  example of each type of data,  your response to this question should furnish a description of how each data type is
  stored and referenced.  
  
  \newpage

\item ({\bf 10 Points}) Given the following variable declarations, answer each question

\hspace*{.25in}
\begin{minipage}{6in}
  \begin{lstlisting}
    int count = 0, value, total;
    final int MAX_VALUE = 100;
    int myValue = 50;
  \end{lstlisting}
\end{minipage}

\begin{enumerate}

\item How many variables are declared?
  \vspace*{1in}

\item What is the type of these declared variables?
  \vspace*{1in}

\item Which of the variables are given an initial value?
  \vspace*{1in}

\item Based on the above declarations, will the following assignment
  statement be accepted by the compiler?  Please fully justify your
  response to this question.


\begin{minipage}{6in}
  \begin{lstlisting}
    myValue = 100;
  \end{lstlisting}
\end{minipage}
  \vspace*{1in}

\item Based on the above declarations, will the following assignment
  statement be accepted by the compiler?  Please fully justify your
  response to this question.
  

\begin{minipage}{6in}
  \begin{lstlisting}
    MAX_VALUE = 50;
  \end{lstlisting}
\end{minipage}
% \vspace*{1in}

\end{enumerate}

\end{enumerate}

\newpage

\noindent
{\bf Section II (Multiple Choice)}

\begin{enumerate}

  \item ({\bf 2 Points}) What description best explains the circumstance in which the ``{\tt else}''
    block will execute when a block of ``{\tt if/else if/else}'' conditional logic executes?

\begin{enumerate}
  \item The {\tt else} block will always execute.

\medskip
\item The {\tt else} block never executes.

\medskip
\item The {\tt else} block executes only when the {\tt if} condition is false.

\medskip 
\item The {\tt else} block executes when one of the {\tt else if} conditions is false.

\medskip 
\item The {\tt else} block executes when at least one of the {\tt else if} conditions is false.

\medskip
\item None of the above; the correct answer is \underline{\hspace{3in}}
\end{enumerate}

% \vspace*{1in}

\item ({\bf 2 Points})
  Suppose that a program running in your terminal seems to be caught in an infinite loop. What command can you
  type to stop this program from executing?
  \begin{enumerate}
    \item {\tt CTRL-Q}
      \medskip 
    \item {\tt gg=G}
      \medskip
    \item {\tt CTRL-c}
      \medskip 
    \item {\tt CTRL-d}
      \medskip
    \item {\tt terminate}
      \medskip
\item None of the above; the correct answer is \underline{\hspace{3in}}
  \end{enumerate}

\newpage

% \bigskip
% \bigskip
% \bigskip
% % \bigskip

\item ({\bf 2 Points})
  Which of the following is needed for a Java program to use the {\tt ArrayList}?
  \begin{enumerate}
    \item {\tt import ArrayList;}
      \medskip 
    \item {\tt import java.list.ArrayList;}
      \medskip 
    \item {\tt insert java.ArrayList;}
      \medskip
    \item {\tt import java.util.ArrayList;}
      \medskip
    \item {\tt load ArrayList;}
      \medskip
    \item None of the above; the correct answer is \underline{\hspace{3in}}
  \end{enumerate}

\bigskip
\bigskip

\item ({\bf 2 Points})
  Which of the following expressions will store the value of 21 in {\tt result}?
  \begin{enumerate}
    \item {\tt result = 6 * ((19 - 4) / 2);}
      \medskip 
    \item {\tt result = 3 * ((19 - 4) / 2);}
      \medskip 
    \item {\tt result = 3 * ((18 - 4) / 2);}
      \medskip 
    \item {\tt result = 3 * ((19 - 8) / 2);}
      \medskip
    \item {\tt result = 4 * ((18 - 4) / 2);}
      \medskip
    \item None of the above; the correct answer is \underline{\hspace{3in}}
  \end{enumerate}

\bigskip
\bigskip

\item ({\bf 2 Points})
  Which of the following statements correctly displays the below 
  message (including spaces separating the words and numbers) when executed:
  \begin{center}
    \verb$The sum of 3 and 6 is 9$
  \end{center}
  \begin{enumerate}
    \item \verb$System.out.println("The sum of" + 3 + "and" + 6 + "is 9");$
    \item \verb$System.out.println("The sum of 3 and 6 is " + (3+6));$
    \item \verb$System.out.println("The sum of " 3 " and " 6 " is " (3+6));$
    \item \verb$System.out.println("The sum of " + 3 + " and " + (3+3) + " is " + (3+3));$
    \item \verb$System.out.println("The sum of " + 3 + " and " + (3+3) + " is " + (3));$
    \item None of the above; the correct answer is \underline{\hspace{3in}}
  \end{enumerate}

\end{enumerate}

\newpage

\noindent
{\bf Part III (Program Output)}

\begin{enumerate}

  \item ({\bf 5 Points}) What is the output from the following Java program?

\hspace*{.25in}
\begin{minipage}{6in}
  \lstset{numbers=left}
  \begin{lstlisting}
    // perform a series of arithmetic computations
    public class BasicComputationsOne
    {
      public static void main(String[] args)
      {
        // declare and initialize the variables
        double a = 4.0;
        int b = 7;
        int c = 5;
        // perform the computation
        a = a - b / 6 * c + b;
        // create the output
        System.out.println(" a = " + a);
      }
    }
  \end{lstlisting}

\end{minipage}

Answer: \line(1,0){250}

\newpage

\item ({\bf 5 Points}) What is the output from the following Java program?

\hspace*{.25in}
\begin{minipage}{6in}
  \lstset{numbers=left}
  \begin{lstlisting}
    public class BasicComputationsTwo
    {
      public static void main(String[] args)
      {
        // declare and initialize the variables
        int a, b, c, d, r;
        a = 1;
        b = 5;
        c = 10;
        d = 15;
        // perform the computation
        r = a - b * c / d + (int)Math.sqrt(a + d);
        // produce the output
        System.out.println(" The answer is: " + r)
      }
    }
  \end{lstlisting}

\end{minipage}

Answer: \line(1,0){250}

\newpage

\item ({\bf 10 Points}) What is the output from the following Java program?

\hspace*{.25in}
\begin{minipage}{6in}
  \lstset{numbers=left}
  \begin{lstlisting}
import java.util.Scanner;
import java.text.DecimalFormat;
public class CircleStatistics
{
   public static void main (String[] args)
   {
      // declare the variables 
      int radius;
      double area, circumference;
      // accept input from the user
      Scanner scan = new Scanner (System.in);
      System.out.print (" Enter the circle's radius: ");
      radius = scan.nextInt();
      // perform the computations
      area = Math.PI * Math.pow(radius, 2);
      circumference = 2 * Math.PI * radius;
      // format and produce the output
      DecimalFormat fmt = new DecimalFormat (" 0.### ");
      System.out.println (" The circle's area: " + 
                          fmt.format(area));
      System.out.println (" The circle's circumference: " +
                           fmt.format(circumference));
   }
}
  \end{lstlisting}

\end{minipage}

Your response to this question should assume that the user inputs the
value of 5 for the radius of the circle. Your answer to this question
can use the fact that the following program produces the output: ``The
value of Pi is: 3.141592653589793''

\hspace*{.25in}
\begin{minipage}{6in}
  \lstset{numbers=left}
  \begin{lstlisting}
public class DetermineTheValueOfPi	
{
    public static void main(String[] args)
    {
        System.out.println(" The value of Pi is: " + Math.PI);
    }
}
  \end{lstlisting}

\end{minipage}

Answer: \line(1,0){250}

\newpage

\item ({\bf 10 Points}) What is the output from the following Java program?

\hspace*{.25in}
\begin{minipage}{6in}
  \lstset{numbers=left}
  \begin{lstlisting}
import java.text.NumberFormat;
import java.util.Scanner;
public class Wages
{
   public static void main (String[] args)
   {
      // declare and initialize the regular pay rate
      final double RATE = 8.25;  
      // declare and initialize the standard hours in a work week
      final int STANDARD = 40;   
      // declare and initialize the total payment variable 
      double pay = 0.0;
      // accept the number of hours that the individual worked
      Scanner scan = new Scanner (System.in);
      System.out.print(" Enter the number of hours worked: ");
      int hours = scan.nextInt();
      System.out.println();

      // pay the worker the correct amount of money
      if (hours > STANDARD) 
      {
         pay = STANDARD * RATE + (hours-STANDARD) * (RATE * 1.5);
      }
      else 
      {
         pay = hours * RATE;
      }

      // correctly format and produce the output
      NumberFormat fmt = NumberFormat.getCurrencyInstance();
      System.out.println (" Gross earnings: " + fmt.format(pay));
   }
}
  \end{lstlisting}

\end{minipage}

Your response to this question should assume that the user inputs the value of 46.

Answer: \line(1,0){250}

\item ({\bf 10 Points}) What is the output from the following Java program?

\hspace*{.25in}
\begin{minipage}{6in}
  \lstset{numbers=left}
  \begin{lstlisting}
    public class Stars
    {
     	public static void main (String[] args)
        {
            final int MAX_ROWS = 10;

            for (int row = 1; row <= MAX_ROWS; row++)
            {
                    for (int star = 1; star <= row; star++)
                    {
                            System.out.print (" * ");
                    }
                    System.out.println();
            }
            for (int row = MAX_ROWS; row >= 0; row--)
            {
                    for (int star = row; star >= 0; star--)
                        {
                            System.out.print (" * ");
                        }
                    System.out.println();
            }
    }
  \end{lstlisting}

\end{minipage}

Answer: \fbox{%
  \parbox{0.8\linewidth}
         {\hspace*{.1in} \newline
\hspace*{.1in} \newline
\hspace*{.1in} \newline
\hspace*{.1in} \newline
\hspace*{.1in} \newline
\hspace*{.1in} \newline
\hspace*{.1in} \newline
\hspace*{.1in} \newline
\hspace*{.1in} \newline
\hspace*{.1in} \newline
\hspace*{.1in} \newline
\hspace*{.1in} \newline
\hspace*{.1in}}
}

\end{enumerate}

\newpage

\noindent
{\bf Section IV (Segments, Methods, Programs)}

\begin{enumerate}

  \item ({\bf 5 Points}) Write a ``program segment'' to set {\tt k} equal to the absolute value of {\tt j} without using
    the absolute value method (i.e., do not use the {\tt Math.abs} provided by Java).

    \vspace*{2in}

  % \item ({\bf 5 Points} Write a method called {\tt findMinimum} that accepts two {\tt int} parameters and returns the
  %     minimum of these two values.

% \item Write a class method called {\tt findMaximum} that accepts two
%   {\tt double} parameters and returns the maximum of these two values.

  \item ({\bf 5 Points}) Write a method named {\tt oddPos} that has one parameter representing an integer {\tt n} and,
    using conditional logic and logical expression(s), returns {\tt true} if {\tt n} is an odd positive integer (e.g.,
      1,3,5,7,\ldots) and {\tt false} otherwise.

    \vspace*{2in}

% \item Write a class method called {\tt readIntArray} that takes one
%   argument {\tt n} and will read and return the {\tt n} elements of an
%   integer array.  After importing the {\tt java.util.Scanner} class at
%   the very top of your program, the class method should use an
%   instance of {\tt java.util.Scanner} to input values from the user.

    \newpage

  \item ({\bf 10 Points}) Write a class that represents a {\tt Parcel} for a delivery service.  Each {\tt Parcel} must
    have a sender, a delivery address, and a tracking number that all should be represented as {\tt String}s.  A {\tt
      Parcel} must also have a delivery status variable of type {\tt boolean}.  Finally, the delivery status of a new
    {\tt Parcel} should always be initialized to {\tt false}.  Write the following methods of this class:

  \begin{enumerate}
    \item A three-parameter constructor that accepts an initial values
      for the name of the sender, the delivery address, and a tracking
      number.

    \item A {\tt delivered} method that changes the delivery status of
      a {\tt Parcel} to {\tt true}.

    \item A zero-parameter {\tt toString} method that returns a {\tt
      String} containing all of the information about a specific {\tt
      Parcel}.  This method should properly format the return value
      with clear labels and include values for the sender, address,
      tracking number, and delivery status.

    \item A {\tt main} method that uses the three-parameter
      constructor to perform the following steps in order: (i) create
      an instance of the {\tt Parcel}, (ii) print out the status of
      the {\tt Parcel} using {\tt println}, (iii) call a method to
      indicate that the {\tt Parcel} has been delivered, (iv) print
      out the status of the {\tt Parcel} using {\tt println}, (v)
      print the date and time on which the program was executed.

  \end{enumerate}

\end{enumerate}

\end{document}
