\documentclass[11pt]{report}
\setlength{\topmargin}{0in}
\setlength{\textheight}{8.5in}
\setlength{\oddsidemargin}{0in}
\setlength{\textwidth}{6.5in}
\usepackage{url}
\usepackage{graphicx}
\usepackage{fancyvrb}
\usepackage{color}
\usepackage{amsmath}
\pagestyle{myheadings}
\markright{\bf Name: \underline{\hspace{2in}}}
\begin{document}
\thispagestyle{empty}
\begin{center}
\bf CMPSC 111\\
Fall 2014\\
October 16, 2014\\
Midterm Exam\\
100 points

\vspace{1in}
Name (printed legibly!): \underline{\hspace{3in}}\\

\bigskip
Signature for Honor Code: \underline{\hspace{3in}}\\
\end{center}

\medskip
\noindent
The quiz is closed book, closed notes, and closed computer. \\
Please place all answers on the test sheets. 

\begin{center}\rule{4in}{1pt} \end{center}

\begin{enumerate}
\item {\bf [3 points]}
What are the input(s), output(s), and behavior(s) of the Java compiler?

\vspace{.75in}
\item {\bf [3 points]}
How many different values can be represented using 7 bits? Why?

\bigskip
\bigskip
\bigskip
\bigskip

\item {\bf [2 points]}
Give examples of at least two Java classes that are part of {\tt java.util} package.
\vspace{.5in}

\item {\bf [2 points]}
  After you use ``{\tt git add}'' for the {\tt Midterm.java} program in your local Git repository, what command(s) do
  you need to use to update your repository on Bitbucket so that it appears on the Bitbucket servers? Write the commands
  as you would type them in your terminal. You may assume that you are in the same directory as the {\tt Midterm.java}
  file.  

% \vspace{.75in}
\newpage

\item {\bf [2 points]}
Which of the following statements correctly displays the following
message (including spaces separating the words and numbers) when executed:
\begin{center}
\verb$The sum of 3 and 6 is 9$
\end{center}
\begin{enumerate}
\item \verb$System.out.println("The sum of" + 3 + "and" + 6 + "is 9");$
\item \verb$System.out.println("The sum of 3 and 6 is " + (3+6));$
\item \verb$System.out.println("The sum of " 3 " and " 6 " is " (3+6));$
\item \verb$System.out.println("The sum of " + 3 + " and " + (3+3) + " is " + (3+3));$
\end{enumerate}

\bigskip

\item {\bf [2 points]}
Which of the following Java expressions has a value equal to 25:
\begin{enumerate}
\item \verb$100 / 2 * 2$
\item \verb$50 % 25$
\item \verb$25 % 50$
\item \verb$25 + 50 / 3$
\end{enumerate}

\bigskip

\item {\bf [3 points]}
What is printed by the following Java statements? 
%Please indicate
%spacing clearly (if you need to, use a symbol like ``\verb*$ $'' to
%indicate a space).
\begin{verbatim}
       int a = 10, b = 7;
       System.out.println("a * b + 5 = " + (a * b) + " + 5");
       System.out.println("\\\\//");
       System.out.println("a + b = " + a + b);
\end{verbatim}

\vspace{1.0in}
\item {\bf [4 points]}
Using correct operator precedence, what values would be printed by the following
print statements?
\begin{verbatim}
System.out.print ( ( 34 - 4 ) \% 4 / ( 12 -10 ) );




\end{verbatim}
\begin{verbatim}
System.out.print ( 34 - 4 \% 4 / 12 - 10); 




\end{verbatim}

\item {\bf [8 points]}
Write a complete Java program that does the following: 
\begin{quote}
Print a message telling the user to type a word\\
Read in a single word typed by the user at the keyboard\\
Print the word back out
\end{quote}
You do \underline{not} need to include 
comments; your program does \underline{not} need to print your 
name and the date. However, all other parts of the program must be
there---the class, the main method, the ``\{'' and ``\}'' characters,
the {\tt import} statements, etc.

Please try to be neat; please try to align things like ``\{'' and ``\}''
characters; please try to clearly distinguish between upper-case and
lower-case letters (so that I will know, for instance, whether
you wrote ``{\tt system}'' with a small ``{\tt s}'' or ``{\tt System}''
with a capital ``{\tt S}'').

\vspace{3in}

\item {\bf [5 points]}
Assume that your terminal screen currently contains the following information
(assume that you are user ``{\tt jjumadinova}''):
\begin{center}
\begin{minipage}{3in}
\begin{verbatim}
jjumadinova@aldenv150 ~/cs111$ ls
lab1     lab2    Template.java
jjumadinova@aldenv150 ~/cs111$ 
\end{verbatim}
\end{minipage}
\end{center}

What Linux commands would you use to perform the following tasks? 
\begin{itemize}
\item create a new directory named {\tt midterm} within your current directory
\item make {\tt midterm} the new current directory (move into directory {\tt midterm})
\item make a copy of {\tt Template.java}, naming the copy {\tt
Midterm.java}
\item compile and run {\tt Midterm} class
\end{itemize}

\vspace{2in}
\newpage

\item {\bf [6 points]}
At one of Meadville's fancy restaurants, a cup of tea costs
\$3. You have a certain number of dollars, stored in an {\tt int}
variable named {\tt cash}. Write the Java statements needed to
print out the maximum number of cups of tea you can purchase,
together with how much cash you will have left after purchasing them.
These should be printed with simple labels, e.g., ``{\tt Cups of tea:}'' 
and ``{\tt Amount left:}'' or similar labels of your choice.

You do not have to write a complete program. You do not need to declare
the variable {\tt cash}---it has already been declared and assigned a
value. All you need to do is determine the two quantities described
above and print them out,
assuming the cost of a cup of tea is \$3.

\vspace{3.5in}

\item {\bf [6 points]}
Define each of the following terms and give examples if appropriate:
\begin{enumerate}
\item compiler

\bigskip
\bigskip
\bigskip
\item primitive data type

\bigskip
\bigskip
\bigskip

\item casting

\bigskip
\bigskip
\bigskip

\item promotion
\bigskip
\bigskip
\bigskip
\end{enumerate}

\item {\bf [5 points]}
Which of the following are allowable as possible variable names in your Java program? Circle all that are allowable.
\begin{itemize}
\item[(a)] iNum
\item[(b)] Grade
\item[(c)] test\#1
\item[(d)] 1st\_rank
\item[(e)] white tiger
\item[(f)] TIGER
\item[(g)] sea\_level\_2014
\item[(h)] Tree(Leaves)
\item[(g)] leavesOnTree
\end{itemize}
\bigskip

\item {\bf [5 points]}
Fill in the missing statement or statements to set 
variable $x$ to a random integer between -5
and 5 (including but not exceeding 5), i.e., a number in the range -5, -4, -3, \ldots , 4, 5,
and set variable $y$ to a random {\tt double} value between -20 and 0 (exclusive, i.e. up to 0 but not including 0).
\begin{Verbatim}[commandchars=\\\{\}]

       Random rand = new Random();
       int x;
       double y;
       \framebox{\rule{4in}{0in}\rule{0in}{2in}}
\end{Verbatim}

\newpage
% \item {\bf [8 points]}
% \label{appproblem}

% Figure \ref{applet} shows a drawing of a house that 
% contains two rectangles and two lines on a white background, and 
% two rectangles filled with yellow color.
% Fill in the body of the {\tt paint} method so that it produces the drawing in
% Figure \ref{applet}. 
% \begin{figure}[htbp]
% \centering
% \begin{tabular}{p{2.5in}p{2in}}
% \begin{minipage}{2.5in}
% \includegraphics[width=2.3in]{graph}
% \end{minipage}
% &

% \begin{minipage}{3.2in}
% You can place your graphic of the house anywhere in the applet, and you may choose
% any sizes for the rectangles, as long as your drawing resembles a house like the one in Figure \ref{applet}. 
% \end{minipage}
% \end{tabular}
% \caption{See problem \ref{appproblem}.}
% \label{applet}
% \end{figure}

% {\bf Answer (provide the body of the {\tt paint} method):}
% \begin{verbatim}
% public void paint(Graphics page)
% {


















% }
% \end{verbatim}

\newpage

\item {\bf [4 points]}
What are the final values of variables {\tt e}, {\tt f}, {\tt g}, and
{\tt h} in the following code?
\begin{verbatim}
       int e = 30, f = 20;
       double g = e / f;
       double h = e / (double) f;
       e = (int) (2*g);
       f = (int) (2*h);
\end{verbatim}
{\bf Answer:}\\
\verb$e = $\underline{\hspace{.5in}}

\bigskip
\verb$f = $\underline{\hspace{.5in}}

\bigskip
\verb$g = $\underline{\hspace{.5in}}

\bigskip
\verb$h = $\underline{\hspace{.5in}}

\bigskip
\item {\bf [8 points]}
Suppose {\tt s} is a {\tt String} variable containing a line of text.
You may assume that the line of text may contain any mixture of
letters, spaces, numbers, punctuation marks, etc. You may also assume
that the string has at least two characters in it.

Complete the Java statements needed to print each of the following
values. If you need
more than one statement, extra room has been provided:
\begin{enumerate}
\item
The length of the string {\tt s}:

\bigskip
\bigskip
\bigskip
\begin{quote}
\verb$System.out.println($\underline{\hspace{3in}}\verb$);$
\end{quote}
\item
The string {\tt s} with all letters changed to upper case:

\bigskip
\bigskip
\bigskip
\begin{quote}
\verb$System.out.println($\underline{\hspace{3in}}\verb$);$
\end{quote}


\item
The middle character of {\tt s} (assume that the middle position is
the length of {\tt s} divided by 2, with truncation if the length is
odd):

\bigskip
\bigskip
\bigskip
\begin{quote}
\verb$System.out.println($\underline{\hspace{3in}}\verb$);$
\end{quote}

\item
The last two characters of {\tt s}:

\bigskip
\bigskip
\bigskip
\begin{quote}
\verb$System.out.println($\underline{\hspace{3in}}\verb$);$
\end{quote}
\end{enumerate}

\bigskip
\item {\bf [8 points]}
Each of the following sequences of Java statements contains an
error. State what the error is.
\begin{enumerate}
\item \mbox{}

\begin{verbatim}
     public static void main(String [ ] args);
     {
          System.out.println("Computer Science is fun!");
     }
     
     
     
     
     
     
\end{verbatim}
\item \mbox{}

\begin{verbatim}
     Scanner scan = new Scanner(System.in);
     String s = scan.nextLine();
     String t = s.length();
     
     
     
     
     
     
\end{verbatim}
\item \mbox{}

\begin{verbatim}
     Random r = new Random;
     
     
     
     
     
     
\end{verbatim}
\item \mbox{}

\begin{verbatim}
     int i = 17 / (double) 10;
     
     
     
     
     
     
\end{verbatim}
\end{enumerate}


\item{\bf [8 points]}
Here is a portion of a Java program:
\begin{verbatim}
        ...
      float num1 = 5.3;
      int num2 = (int) num1 % 2;
      int result = ++num2;
      System.out.printf("Number 1 is %-10f %n Number 2 is %-10d %n 
      					Result is %-10d ", num1, num2, result);
        ...
\end{verbatim}
What output does it produce? You can indicate spaces using a character like ``\verb*$ $''.

\vspace{1.2in}
\item {\bf [8 points]}
Assume that a {\tt DecimalFormat} class has been imported and assume that variable named {\tt num} of type {\tt double} has been input by the user. Write Java statements to calculate the square root of {\tt num} and print it with 2 decimal places using an appropriate method from the {\tt DecimalFormat} class. Remember, you will need to create an instance of the {\tt DecimalFormat} (declare and initialize). 
\begin{verbatim}
       ...
      double num = scan.nextDouble();
      // your statements
      
      
      
      
      
      
      
      
      
      
      
   

\end{verbatim}

\end{enumerate}


 {\bf Please attach your notes (if you have used any) to the test before you submit it.}
 
 
\end{document}
