%!TEX root=cs111F2014-fp.tex 
% mainfile: cs111F2014-fp.tex 

%!TEX root=cs112S2014-lab1.tex
% mainfile: cs112S2014-lab1.tex 
% Typical usage (all UPPERCASE items are optional):
%       \input 580pre
%       \begin{document}
%       \MYTITLE{Title of document, e.g., Lab 1\\Due ...}
%       \MYHEADERS{short title}{other running head, e.g., due date}
%       \PURPOSE{Description of purpose}
%       \SUMMARY{Very short overview of assignment}
%       \DETAILS{Detailed description}
%         \SUBHEAD{if needed} ...
%         \SUBHEAD{if needed} ...
%          ...
%       \HANDIN{What to hand in and how}
%       \begin{checklist}
%       \item ...
%       \end{checklist}
% There is no need to include a "\documentstyle."
% However, there should be an "\end{document}."
%
%===========================================================
\documentclass[11pt,twoside,titlepage]{article}
%%NEED TO ADD epsf!!
\usepackage{threeparttop}
\usepackage{graphicx}
\usepackage{latexsym}
\usepackage{color}
\usepackage{listings}
\usepackage{fancyvrb}
%\usepackage{pgf,pgfarrows,pgfnodes,pgfautomata,pgfheaps,pgfshade}
\usepackage{tikz}
\usepackage[normalem]{ulem}
\tikzset{
    %Define standard arrow tip
%    >=stealth',
    %Define style for boxes
    oval/.style={
           rectangle,
           rounded corners,
           draw=black, very thick,
           text width=6.5em,
           minimum height=2em,
           text centered},
    % Define arrow style
    arr/.style={
           ->,
           thick,
           shorten <=2pt,
           shorten >=2pt,}
}
\usepackage[noend]{algorithmic}
\usepackage[noend]{algorithm}
\newcommand{\bfor}{{\bf for\ }}
\newcommand{\bthen}{{\bf then\ }}
\newcommand{\bwhile}{{\bf while\ }}
\newcommand{\btrue}{{\bf true\ }}
\newcommand{\bfalse}{{\bf false\ }}
\newcommand{\bto}{{\bf to\ }}
\newcommand{\bdo}{{\bf do\ }}
\newcommand{\bif}{{\bf if\ }}
\newcommand{\belse}{{\bf else\ }}
\newcommand{\band}{{\bf and\ }}
\newcommand{\breturn}{{\bf return\ }}
\newcommand{\mod}{{\rm mod}}
\renewcommand{\algorithmiccomment}[1]{$\rhd$ #1}
\newenvironment{checklist}{\par\noindent\hspace{-.25in}{\bf Checklist:}\renewcommand{\labelitemi}{$\Box$}%
\begin{itemize}}{\end{itemize}}
\pagestyle{threepartheadings}
\usepackage{url}
\usepackage{wrapfig}
% removing the standard hyperref to avoid the horrible boxes
%\usepackage{hyperref}
\usepackage[hidelinks]{hyperref}
% added in the dtklogos for the bibtex formatting
\usepackage{dtklogos}
%=========================
% One-inch margins everywhere
%=========================
\setlength{\topmargin}{0in}
\setlength{\textheight}{8.5in}
\setlength{\oddsidemargin}{0in}
\setlength{\evensidemargin}{0in}
\setlength{\textwidth}{6.5in}
%===============================
%===============================
% Macro for document title:
%===============================
\newcommand{\MYTITLE}[1]%
   {\begin{center}
     \begin{center}
     \bf
     CMPSC 112\\Introduction to Computer Science II\\
     Spring 2014
     \medskip
     \end{center}
     \bf
     #1
     \end{center}
}
%================================
% Macro for headings:
%================================
\newcommand{\MYHEADERS}[2]%
   {\lhead{#1}
    \rhead{#2}
    %\immediate\write16{}
    %\immediate\write16{DATE OF HANDOUT?}
    %\read16 to \dateofhandout
    \def \dateofhandout {January 21, 2014}
    \lfoot{\sc Handed out on \dateofhandout}
    %\immediate\write16{}
    %\immediate\write16{HANDOUT NUMBER?}
    %\read16 to\handoutnum
    \def \handoutnum {2}
    \rfoot{Handout \handoutnum}
   }

%================================
% Macro for bold italic:
%================================
\newcommand{\bit}[1]{{\textit{\textbf{#1}}}}

%=========================
% Non-zero paragraph skips.
%=========================
\setlength{\parskip}{1ex}

%=========================
% Create various environments:
%=========================
\newcommand{\PURPOSE}{\par\noindent\hspace{-.25in}{\bf Purpose:\ }}
\newcommand{\SUMMARY}{\par\noindent\hspace{-.25in}{\bf Summary:\ }}
\newcommand{\DETAILS}{\par\noindent\hspace{-.25in}{\bf Details:\ }}
\newcommand{\HANDIN}{\par\noindent\hspace{-.25in}{\bf Hand in:\ }}
\newcommand{\SUBHEAD}[1]{\bigskip\par\noindent\hspace{-.1in}{\sc #1}\\}
%\newenvironment{CHECKLIST}{\begin{itemize}}{\end{itemize}}


\usepackage[compact]{titlesec}

\begin{document} \MYTITLE{Final Project: Real-World Applications of Computer Science} 
\MYHEADERS{Final Project}{Due: Friday, December 11, 2014 at 5:00 pm}

\vspace*{-.2in}

\section*{Introduction}

Throughout the semester, you have learned more about the basics of computer science and Java programming by studying, in
a hands-on fashion, topics such as data and expressions, the use and creation of Java classes, conditionals and loops,
and arrays.  This final project invites you to explore, in greater detail, a real-world application of computer science.
You will learn more about how to use, implement, test, and evaluate different types of real-world computer software.
Since you will complete the final project with a partner, you will also learn more about how the Git version control
system can support collaborative software development.

Your project should result in a detailed report that includes all of your source code, in addition to written materials
and technical diagrams that highlight the key contributions of your work.  The report should include a description of
why the chosen topic is important and discuss the implementation and/or experimentation that you undertook.  The written
material should be precise, formal, appropriately formatted, grammatically correct, informative, and interesting.  The
source code that you write must be carefully documented and tested.  If you install and use existing computer software
(e.g., a Java library for natural language processing),  the steps for installation and use should be clearly
documented in your report. Also, the report must explain the steps to run your own Java program.
Finally, the report must detail the work completed by each member of your partnership; individual contributions should
also be reflected in the Git repository logs.

\section*{Description of the Topics}

Each partnership is invited to pick one of the following projects.  Please note that a partnership selecting the
student-designed project must first discuss their idea with the instructor, during today's laboratory session, and
receive feedback and then final approval.  Please note that you are fully responsible for ensuring the feasibility of
the project that you propose.

\begin{enumerate}

  \item {\bf Cryptography and Cryptanalysis}: Explore a topic in the fields that make up the ``art and science of
    sending and decoding secret messages''. This project invites you and your partner to implement and test several cryptography
    and/or cryptanalysis systems.  To start, you should investigate, implement, and test ciphers such as the Caesar and
    Vigenere ciphers. Then, you should use your ciphers to demonstrate that you can successfully send secret messages
    by, for instance, email messages. In addition to creating and testing these Java programs, your report should
    include a detailed discussion of how your chosen algorithms work.

  \item {\bf Computer Graphics}: Potentially using your textbook's sections on computer graphics as a starting point,
    implement a complete program that displays computer graphics.  Students who select this project should consider ways
    in which the graphics can, for instance, represent realistic entities, support interactivity, and properly adhere to
    artistic principles of color, light, and layout. In addition to furnishing the project's source code, your report
    should include a detailed artistic, technical, and mathematical commentary of the final graphics.

\end{enumerate} 

\section*{Final Project Deadlines}

This assignment invites you to submit printed and signed versions of the following deliverables: 

\vspace*{-.05in}
\begin{enumerate}

  \itemsep0in

  \item {\bf Project Assigned and Project Proposal:} Thursday, November 20, 2014

    After meeting with the course instructor and your partner, pick a topic for your final project.  Remember, if your
    team selects the student-designed project, you must first have your project verified by the course instructor.  Next, make
    sure that you create a Git repository that can be accessed by the instructor. Finally, write and submit a one-page
    proposal for your project. While you can use the project descriptions on the previous pages as a starting point,
    your proposal should have an informative title, an abstract, a description of the main idea, a plan for completing
    the work, and an initial listing of the tasks that you must complete.

  \item {\bf Status Update and Project Demonstration}: Thursday, December 4, 2014

    As you continue working on your project, please submit a one paragraph status update in printed form and through
    your Git repository.  In addition, you should give a demonstration, during the laboratory session, highlighting the
    most important part of your system that you have finished implementing. For instance, if you decide to create a
    benchmarking framework for XML compression algorithms, then you could show how to configure and use the framework,
    create data sets, and/or visualize the empirical results.

  \item {\bf Final Project Due Date}: Thursday, December 11, 2014 at 5 pm

    You should submit the final version of your project, in printed form and the Git repository. This submission should
    include all of the relevant source code and output, the written report, and any additional materials that will
    demonstrate the success of your project.  While you are encouraged to turn in the final project before the final
    examination starts on the due date, students must submit the completed assignment before 5 pm on the due date.

\end{enumerate}
\vspace*{-.05in}

% \noindent In adherence to the Honor Code, students should complete this assignment individually. While it is appropriate
% for students in this class to have high-level conversations about the assignment with other class members, it is
% necessary to distinguish carefully between an individual who discusses the principles underlying a problem with others
% and the student who produces an assignment that is identical to, or merely a variation on, the work of someone else.  As
% such, deliverables that are nearly identical to the work of others will be taken as evidence of violating the
% \mbox{Honor Code}.  Students should contact the course instructor with questions about this course policy.

In adherence to the Honor Code, students should complete this assignment while exclusively collaborating with the
other member of their team. While it is appropriate for students in this class---who are not in the same team---to have
high-level conversations about the assignment, it is necessary to distinguish carefully between the team that discusses
the principles underlying a problem with another team and the team that produces an assignment that is identical to, or
merely a variation on, the work of another team.  Deliverables from one team that are nearly identical to the work of
another team will be taken as evidence of violating Allegheny College's \mbox{Honor Code}.

  \end{document}
