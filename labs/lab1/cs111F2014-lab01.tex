%!TEX root=cs111F2014-lab01.tex
% mainfile: cs111F2014-lab01.tex

\input{111pre}
\begin{document}
\MYTITLE{Lab 1 \\ Assigned: Thursday, 4 September 2014 \\ Due: Thursday, 11 September 2014, 2:30 pm}

\subsection*{Summary}

Learn how to get around in the Ubuntu Linux operating system and how to create, compile, and execute simple Java
programs using the powerful ``{\tt gvim}'' text editor and the Ubuntu terminal.

\vspace*{-.1in}
\subsection*{General Guidelines}

\begin{itemize}
  \setlength{\itemsep}{0pt}

  \item {\bf Work on the Alden Hall computers.} If you want to work on a different machine, be sure to transfer your
    programs to the Alden machines and re-run them before submitting. However, please remember that, as described on the
    syllabus, students should complete assignments using the specialized workstations in the Alden Hall laboratories;
    the course instructor and the teaching assistants normally are not available to help students configure their own
    computers.

  \item {\bf Keep all of your files!} Don't delete your programs and reports after you hand them in---you will need
    them again later when you study for the quizzes and examinations and work on the other laboratory, practical, and
    final project assignments.

  \item {\bf Back up your files regularly}. Use a flash drive, Google Drive, or your favorite backup method to keep a
    copy of your files in reserve. In the event of a system failure, students are responsible for ensuring that they
    have access to a recent backup copy of all their files.

  \item {\bf Review the Honor Code policy on the syllabus.} Remember that you may discuss programs with others, but
    copying programs is a violation of the College's Honor Code.

\end{itemize}

\vspace*{-.3in}
\subsection*{Your Account in Alden Hall}

In advance of today's lab you have already received the details about your Alden Hall computer account and learn how to
log on, change your password, and log out.  You should ensure that you have recorded how to complete these steps in your
notebook; please report any problems as soon as they occur. You may use this account on any computer in Alden labs 101,
103, or 109. Your files are stored on a central server; you don't have to use the same machine every time you log on.

Hours of lab availability are posted on the bulletin board in each lab and on the following Web site:
\url{http://www.cs.allegheny.edu/labs/}; the on duty lab monitor is available in Alden 101.

\vspace*{-.1in}
\subsection*{Creating Your First Java Program}

In order to create a program, you need a {\em text editor}. There are many different text editors on your workstation, and
you should feel free to explore these on your own. Since it is a powerful text editor known for helping computer
scientists ``edit text at the speed of thought'', in this class we will use the text editor called ``{\tt gvim}''.

% : \raisebox{-1.5em}{\includegraphics[width=.7in]{images/terminal}}

There are several ways to launch {\tt gvim}, but for today please use a method described in this paragraph.  On the
left side of your screen, click on the icon that contains the ``{\tt >}'' symbol.  Alternatively, you can type the
``Super'' key, start typing the word ``terminal'', and then select that program.  Another way to open a terminal
involves typing the key combination {\tt <Ctrl>-<Alt>-t}.

% (If you don't see it, right-click on the desktop and choose ``Open in Terminal'' from the menu.)  This opens a {\em
%   terminal} window, a plain window with text. For today, you'll work with the terminal window rather than with folders,
% icons, and the mouse.

Now, you should type the following commands {\em exactly as shown} into the terminal window.  (The ``{\tt 1}'' is the digit ``one'', not
  the letter ``ell.'') All Linux, Java, and {\tt gvim} commands are case-sensitive, so be sure to capitalize the file
name ``{\tt Lab1.java}'' but nothing else.  Don't worry if you make a mistake---just ask the course instructor or a
teaching assistant for help and then try again.

\begin{verbatim}
       mkdir cs111F2014
       cd cs111F2014
       mkdir labs
       cd labs
       mkdir lab1
       cd lab1
       gvim Lab1.java
\end{verbatim}

       At this point, a new window should open up. This is the {\tt gvim} editor.
       You may notice that if you start typing, nothing appears (unless you happen to
         hit certain letters such as ``{\tt i},'' ``{\tt o}'', ``{\tt a}'', and a few
         others). This is because you are not in ``insert mode.'' To get into insert mode,
       just type the letter ``{\tt i}'' (lower case). Once you do this, the window should 
       look like the one in Figure \ref{gvim-insert}. Note the word ``{\tt --INSERT--}''
       in the lower left corner!

       \begin{figure}[htbp]
         \centering
         \includegraphics[width=4.5in]{images/gvim-insert}
         \caption{Insertion mode in {\tt gvim}}
         \label{gvim-insert}
       \end{figure}

       Type the program from Figure
       \ref{lab1prog} into the window, {\em substituting your actual name for the words
         ``Your Name''!} When you are finished, press the ESC key (upper left corner).
       This should remove the word ``{\tt --INSERT--}'' from the bottom of the screen 
       and take you out of insert mode.

       \begin{figure}[htbp]
         \centering
         \includegraphics[width=5.8in]{images/lab1prog}
         \caption{First program}
         \label{lab1prog}
       \end{figure}

       Use the ``File/Save'' command to save your program.
       %named ``{\tt cs111/lab1}'' (see Figure \ref{save}).
       %
       %\begin{figure}[htbp]
       %\centering
       %\includegraphics[width=3in]{images/emacssaveas}
       %\caption{Saving the program in the {\tt lab1} folder}
       %\label{save}
       %\end{figure}

       Go back to your terminal window. (You can leave the {\tt gvim} window open.)
       Type the following command at the
       prompt---this is the ``compile'' step:
       \begin{quote}
         \verb$javac Lab1.java$
       \end{quote}
       If you get any error messages, go back into {\tt gvim} and try to figure 
       out what you mis-typed and fix it. Re-save your program  and 
       re-compile the program (i.e., re-run the ``{\tt javac}'' command).

       When all errors are eliminated, type the following in the terminal
       window---this is the ``execute'' step:
       \begin{quote}
         \verb$java Lab1$
       \end{quote}
       You should see your name, today's date, the lab number, and two more
       lines of text. Make sure there are spaces separating words in your output
       (did you forget to put a space inside the quotation marks after your
         last name?). If not, repair the program and re-compile and re-run it.
       %
       %\begin{figure}[htbp]
       %\centering
       %\includegraphics[width=3.5in]{images/lincolnrun}
       %\caption{Successful compilation and execution}
       %\label{output}
       %\end{figure}

       Now print out your program
       directly from {\tt gvim}. Do this with the ``File/Print'' menu item. Pick up
       your output at one of the two printers in the front of the lab (101a or 101b).
       %we'll introduce the
       %``Libre Writer'' application (which you may find useful for other
       %purposes).
       %On the menu on the left, run
       %the ``Libre Writer'' application. Use the ``Insert/File'' menu item to
       %import your {\tt Lincoln.java} program into the document; use the font
       %named ``Courier 10 Pitch'' so that your text appears as fixed-width.
       %
       %Then go back to your terminal window and, with the mouse, highlight
       %everything from the line containing the ``{\tt javac}'' command through
       %the line following your last output line. Copy this (right-click, choose
       %``Copy'') and then paste it at the bottom of your Libre office file.
       %Add a line labeling the program output.
       %
       %The result should look something like Figure \ref{libre}.
       %
       %\begin{figure}[htbp]
       %\centering
       %\includegraphics[width=5.5in]{images/libre}
       %\caption{The Lab Report in Libre Writer}
       %\label{libre}
       %\end{figure}
       %
       %Save your report in your {\tt cs111/lab1} folder with a name like ``{\tt
       %yourname-lab1report.odt}'' (the ``{\tt .odt}'' should be automatically
       %supplied).

       %Now open the Firefox browser and go to {\tt sakai.allegheny.edu}. Log on
       %and go to your ``Drop box''. Under the ``Add'' menu, choose ``Upload Files''.
       %Upload your {\tt Lincoln.java} file {\em AND} your report. (Yes, I want
       %the program uploaded twice, once in the report and once by itself.)
       %{\color{red}\bf[Why?]}

       %Finally, print out your report. You can do this with the ``File/Print''
       %command in Libre Writer.

       Sign your name at the top of your printout---this is the Honor Code pledge
       that the work you are handing in was done according to the Honor Code guidelines.

       \subsection*{Write Your Own Program}
       Imitate the steps from before, using a different program name (for
         example, ``{\tt Lab1Part2.java}''). Note that the name in your ``{\tt public
         class}'' statement must exactly match the portion of the file name
       preceding the ``{\tt .java}''.

       Print out something different (but still include your name and the date
         as shown in the example). Experiment. Try printing several more lines of output.
       Try making errors (for instance, omitting the ``{\tt ;}'' or capitalizing
         something incorrectly or other mistakes). This is how you learn---by trying
       things. Ask questions---lots of them! 

       \subsection*{Deliverable}
       Hand in your stapled printed programs (there should be
         at least two, but you might do several more if you want) before the due date
       and time. 
       This is the only time when you will have to hand in a printed version of your assignment.

       \end{document}
