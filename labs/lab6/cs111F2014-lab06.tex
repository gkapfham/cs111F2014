%!TEX root=cs111F2014-lab06.tex
% mainfile: cs111F2014-lab06.tex

%!TEX root=cs111F2014-lab09.tex
% mainfile: cs111F2014-lab09.tex

% CS 111 style
% Typical usage (all UPPERCASE items are optional):
%       \input 111pre
%       \begin{document}
%       \MYTITLE{Title of document, e.g., Lab 1\\Due ...}
%       \MYHEADERS{short title}{other running head, e.g., due date}
%       \PURPOSE{Description of purpose}
%       \SUMMARY{Very short overview of assignment}
%       \DETAILS{Detailed description}
%         \SUBHEAD{if needed} ...
%         \SUBHEAD{if needed} ...
%          ...
%       \HANDIN{What to hand in and how}
%       \begin{checklist}
%       \item ...
%       \end{checklist}
% There is no need to include a "\documentstyle."
% However, there should be an "\end{document}."
%
%===========================================================
\documentclass[11pt,twoside,titlepage]{article}
%%NEED TO ADD epsf!!
\usepackage{threeparttop}
\usepackage{graphicx}
\usepackage{latexsym}
\usepackage{color}
\usepackage{listings}
\usepackage{fancyvrb}
%\usepackage{pgf,pgfarrows,pgfnodes,pgfautomata,pgfheaps,pgfshade}
\usepackage{tikz}
\usepackage[normalem]{ulem}
\tikzset{
    %Define standard arrow tip
%    >=stealth',
    %Define style for boxes
    oval/.style={
           rectangle,
           rounded corners,
           draw=black, very thick,
           text width=6.5em,
           minimum height=2em,
           text centered},
    % Define arrow style
    arr/.style={
           ->,
           thick,
           shorten <=2pt,
           shorten >=2pt,}
}
\usepackage[noend]{algorithmic}
\usepackage[noend]{algorithm}
\newcommand{\bfor}{{\bf for\ }}
\newcommand{\bthen}{{\bf then\ }}
\newcommand{\bwhile}{{\bf while\ }}
\newcommand{\btrue}{{\bf true\ }}
\newcommand{\bfalse}{{\bf false\ }}
\newcommand{\bto}{{\bf to\ }}
\newcommand{\bdo}{{\bf do\ }}
\newcommand{\bif}{{\bf if\ }}
\newcommand{\belse}{{\bf else\ }}
\newcommand{\band}{{\bf and\ }}
\newcommand{\breturn}{{\bf return\ }}
\newcommand{\mod}{{\rm mod}}
\renewcommand{\algorithmiccomment}[1]{$\rhd$ #1}
\newenvironment{checklist}{\par\noindent\hspace{-.25in}{\bf Checklist:}\renewcommand{\labelitemi}{$\Box$}%
\begin{itemize}}{\end{itemize}}
\pagestyle{threepartheadings}
\usepackage{url}
\usepackage{wrapfig}
\usepackage{hyperref}
%=========================
% One-inch margins everywhere
%=========================
\setlength{\topmargin}{0in}
\setlength{\textheight}{8.5in}
\setlength{\oddsidemargin}{0in}
\setlength{\evensidemargin}{0in}
\setlength{\textwidth}{6.5in}
%===============================
%===============================
% Macro for document title:
%===============================
\newcommand{\MYTITLE}[1]%
   {\begin{center}
     \begin{center}
     \bf
     CMPSC 111\\Introduction to Computer Science I\\
     Fall 2014\\
     %Janyl Jumadinova\\
     %\url{http://cs.allegheny.edu/~jjumadinova/111}
     \medskip
     \end{center}
     \bf
     #1
     \end{center}
}
%================================
% Macro for headings:
%================================
\newcommand{\MYHEADERS}[2]%
   {\lhead{#1}
    \rhead{#2}
    \immediate\write16{}
    \immediate\write16{DATE OF HANDOUT?}
    \read16 to \dateofhandout
    \lfoot{\sc Handed out on \dateofhandout}
    \immediate\write16{}
    \immediate\write16{HANDOUT NUMBER?}
    \read16 to\handoutnum
    \rfoot{Handout \handoutnum}
   }

%================================
% Macro for bold italic:
%================================
\newcommand{\bit}[1]{{\textit{\textbf{#1}}}}

%=========================
% Non-zero paragraph skips.
%=========================
\setlength{\parskip}{1ex}

%=========================
% Create various environments:
%=========================
\newcommand{\PURPOSE}{\par\noindent\hspace{-.25in}{\bf Purpose:\ }}
\newcommand{\SUMMARY}{\par\noindent\hspace{-.25in}{\bf Summary:\ }}
\newcommand{\DETAILS}{\par\noindent\hspace{-.25in}{\bf Details:\ }}
\newcommand{\HANDIN}{\par\noindent\hspace{-.25in}{\bf Hand in:\ }}
\newcommand{\SUBHEAD}[1]{\bigskip\par\noindent\hspace{-.1in}{\sc #1}\\}
%\newenvironment{CHECKLIST}{\begin{itemize}}{\end{itemize}}

\begin{document}
\MYTITLE{Lab 2 for Sections 03 and 04\\9 October 2014\\Due Thursday, 16 October by 2:30 pm}
%\MYHEADERS{Lab 2}{Due 12 Sep, 2pm}

\subsection*{Objectives}

In this laboratory assignment, you will learn more about using the {\tt java.lang.Math} class to perform numerical
calculations, further explore the creation of formatted output, learn how to use enumerated types, and practice calling
methods in another Java class.  Additionally, since real-world software developers often have to debug source code
created by other developers and add features to existing code, you will participate in a ``bug hunt'' and add new
source code to an existing system. Ultimately, you will create a working program that comprised of two Java classes.

\subsection*{General Guidelines for Labs}

\begin{itemize}
\item
{\bf Work on the Alden Hall computers.} If you want to work on a different
machine, be sure to transfer your programs to the Alden
machines and re-run them before submitting.
\item
{\bf Update your repository often!} You should add, commit, 
and push your updated files each time you work on them.  I will not grade 
your programs until the due date has passed.
\item
{\bf Review the Honor Code policy.} You
may discuss programs with others, but programs that are nearly identical
to others will be taken as evidence of violating the Honor Code.
\end{itemize}

\subsection*{Reading Assignment}

Please review the handout on ``Tips on Using Linux and Gvim'' (available in the shared repository in the ``handouts''
  directory).  Also review lecture slides and Sections 2.2--2.6 in your textbook.

\subsection*{Create a Program Template}

\vspace*{-.1in}
\subsection*{Write a Java Program to Perform a Simple Calculation}

\subsection*{Required Deliverables}

This assignment invites you to submit electronic versions of the following deliverables through the Bitbucket repository
that you created during the first practical assignment.  As you complete this step, you should make sure that you
created a {\tt lab2/} directory within the Git repository.  Then, you can save all of the required deliverables in the
{\tt lab2/} directory---please see the course instructor or a teaching assistant if you are not able to create your
directory properly.  Additionally, students should submit signed and printed versions of all the required deliverables.

\begin{enumerate}

	\item A completed, properly commented and formatted {\tt Lab2.java} program.

        \item The output from running {\tt Lab2} in the terminal window. You may use {\tt gvim} to save your output as
          follows: using the mouse, select everything from the ``{\tt java Lab2}'' command to the end of your output.
          Right-click on the selected text and copy it.  Type ``{\tt gvim output}''---note that this {\em not} a Java
          program!---and use the ``Edit/Paste'' menu item to paste your program's output into the file.  Now, use ``{\tt
          :w}'' or the ``File/Save'' menu item to save this file.

\end{enumerate}

Share your program and the output file with me through your Git repository by correctly using ``{\tt git add}'', ``{\tt
git commit}'', and ``{\tt git push}'' commands. When you are done, please ensure that the Bitbucket Web site has
a {\tt lab2/} directory in your repository with the two files called {\tt Lab2.java} and {\tt output}. You should see 
the instructor if you have questions about assignment submission.

In adherence to the Honor Code, students should complete this assignment on an individual basis. While it is appropriate
for students in this class to have high-level conversations about the assignment, it is necessary to distinguish
carefully between the student who discusses the principles underlying a problem with others and the student who produces
assignments that are identical to, or merely variations on, someone else's work.  With the exception of the source code
that was provided in the Git repository, deliverables that are nearly identical to the work of others will be taken as
evidence of violating the \mbox{Honor Code}.  

\end{document}
