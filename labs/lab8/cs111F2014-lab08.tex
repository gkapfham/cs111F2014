%!TEX root=cs111F2014-lab08.tex
% mainfile: cs111F2014-lab08.tex

\input{111pre}
\begin{document}
\MYTITLE{Lab 8 for Sections 03 and 04 \\ 30 October 2014\\
Due Thursday, 6 November by 2:30pm }

\vspace{-0.2in}
\subsection*{Objectives}
\vspace{-0.05in}

To enhance your experience with designing and implementing your own Java classes, including the completion of tasks such
picking the right the instance variables, and creating both the constructors and the methods.  Additionally, to practice
using compound {\tt if/else} statements as part of a solution to a real-world problem involving a popular and well-known
game.

\vspace{-0.15in}
\subsection*{General Guidelines for Labs}
\vspace{-0.05in}
\begin{itemize}
\item
{\bf Work on the Alden Hall computers.} If you want to work on a different
machine, be sure to transfer your programs to the Alden
machines and re-run them before submitting.
\item
  {\bf Update your repository often!} You should {\tt add}, {\tt commit}, 
  and {\tt push} your updated files each time you work on them.  I will not grade 
your programs until the due date has passed.
\item
{\bf Review the Honor Code policy.} You
may discuss programs with others, but programs that are nearly identical
to others will be taken as evidence of violating the Honor Code.
\end{itemize}

\vspace{-0.15in}
\subsection*{Reading Assignment}
\vspace{-0.05in}

To learn more about {\tt if} statement and boolean expressions, review Sections 5.1--5.3, and to review material on
constructing classes and methods, read Sections 4.1--4.5.  Students who are not familiar with the Sudoku grame are
encouraged to review \url{http://en.wikipedia.org/wiki/Sudoku}.

\vspace{-0.1in}
\subsection*{Create a new directory and two Java programs}
\vspace{-0.05in}
In your own {\tt cs111F2014-<your user name>} repository inside {\tt labs} directory, create a directory called {\tt lab8}. Type ``{\tt cd lab8}'' to move to the new directory. Then using {\tt gvim} create a {\tt Lab8.java} and {\tt Lab8Main.java} files.

\vspace{-0.1in}
\subsection*{Sudoku Checker}
\vspace{-0.05in}
Sudoku is a logic-based placement puzzle. The aim of this puzzle is to enter a numerical digit from 1 through 9 in each cell of a 9×9 grid made up of 3×3 subgrids (called ``regions''), starting with various digits given in some cells (the ``givens''). Each row, column, and region must contain only one instance of each numeral.

\noindent For this lab, you are going to write a Sudoku validator for 4x4 grids made up of 2x2 regions instead of 9x9 grids made up of 3x3 regions.  Sudoku puzzles that use 4x4 grids only use the numerical digits 1 through 4 instead of 1 through 9.

\noindent Here are two correct 4x4 Sudoku grids:\\
\includegraphics[scale=0.3]{grids}

\noindent Notice how each row, column and region has the individual numbers from 1 through 4 used only once.  This means that the values used in each row, column and region must add up to 10 (1 + 2 + 3 + 4).  This is how we will check our 4x4 Sudoku grids for this assignment.

\noindent Allow your user to enter their Sudoku grids row-by-row, separating each of the four values by a space and hitting ENTER at the end of the row.

\noindent When the user has entered all 16 values into the program, you should output validation checks for each region, row and column.  An example run can be seen below.
\begin{verbatim}
Welcome to the Sudoku Checker v1.0!

This program checks simple, small, 4x4 Sudoku grids for correctness. 
Each column, row and 2x2 region contains the numbers 1 through 4 only once.

To check your Sudoku, enter your board one row at a time, 
with each digit separated by a space.  Hit ENTER at the end of a row.

Enter Row 1: 3 2 4 1
Enter Row 2: 4 1 3 2
Enter Row 3: 1 4 2 3
Enter Row 4: 2 3 1 4

REG-1:GOOD
REG-2:GOOD
REG-3:GOOD
REG-4:GOOD

ROW-1:GOOD
ROW-2:GOOD
ROW-3:GOOD
ROW-4:GOOD

COL-1:GOOD
COL-2:GOOD
COL-3:GOOD
COL-4:GOOD

SUDO:VALID
\end{verbatim}

\noindent Your program does not have to check for the numbers 1 through 4 being used uniquely in each row, column and region.  In other words, don't worry about validating their input to make sure they only enter the digits 1, 2, 3 or 4 and only enter them once per row, column and region.  
You simply need to check that each row, column and region adds up to 10. This simplistic type of checking will, however, allow some bad Sudoku grids to be validated as good. This is ok. 

\noindent Rows, columns and regions are identified as found below:\\
\includegraphics[scale=0.5]{sud}

\subsubsection*{{\tt Lab8Main} Class}

To test your implementation of the {\tt Lab8} class, use the following code for Lab8Main.java.
\begin{verbatim}
public class Lab8Main
{
     public static void main ( String args[] )
     {
          Lab8 foo = new Lab8();
          foo.getGrid();
          foo.checkGrid();
     }
}
\end{verbatim}

\subsubsection*{ {\tt Lab 8} Class}
\noindent The basic structure of {\tt Lab8.java} is as follows:
\begin{verbatim}
import java.util.Scanner;

public class Lab8
{

     // put private data members here

     // put constructor here

     // put getGrid() here

     // put checkGrid() here
}
\end{verbatim}
The UML diagram for {\tt Lab8} is:
\begin{tabular}{|l|}
\hline
\textbf{ Lab8} \\
\hline
- w1 : Integer\\
- w2 : Integer\\
- w3 : Integer\\
- w4 : Integer\\
- x1 : Integer\\
- x2 : Integer\\
- x3 : Integer\\
- x4 : Integer\\
- y1 : Integer\\
- y2 : Integer\\
- y3 : Integer\\
- y4 : Integer\\
- z1 : Integer\\
- z2 : Integer\\
- z3 : Integer\\
- z4 : Integer\\
\hline
$<<constructor>>$ Lab8 ( )\\
\hline
+ getGrid ( )\\
+ checkGrid ( )\\
\hline
\end{tabular}

\noindent What follows is a short description of what each data member represents and what each method does:\\
\begin{tabular}{|l|p{10cm}|}
\hline
{\tt w1, w2, w3, w4} &  \\
{\tt x1, x2, x3, x4} & \\
{\tt y1, y2, y3, y4} & \\
{\tt z1, z2, z3, z4}	&  Private data members that will store the numbers the user inputs.  {\tt w1} through {\tt w4} are for the first row, {\tt x1} through {\tt x4} are for the second row, etc.  Do not have any other private data members in your class.\\
\hline
Lab8()	&This is the constructor.  It should display welcome greetings, explaining the rules of the 4x4 Sudoku grid we're processing.\\
\hline
getGrid()	& This public method should read the Sudoku grid from the user row by row, using spaces between values in each row.\\
\hline 
checkGrid() &	This public method uses the private data members to test whether or not the given values are a valid Sudoku grid.\\
& Your method does not have to check for the numbers 1 through 4 being used uniquely in each row, column and region.  In other words, don't worry about validating their input to make sure they only enter the digits 1, 2, 3 or 4 and only enter them once per row, column and region. \\ 
& When producing your output, you must first validate the regions, then the rows and then the columns.  Each region, row and column validation should appear on a line by itself. \\
\hline
\end{tabular}

\vspace{-0.05in}
\subsection*{Points to Think About}
\vspace{-0.05in}
Since we have not talked about programming concepts such as arrays yet (you can not use them), you will need to input your 16 values into 16 separate variables.  Some suggestions would be a1, a2, a3 and a4 for the first row and then b1, b2, b3 and b4 for the second row, and so on.
Once you validate that one row is good or not good, the rest of the program should essentially be a matter of copy and paste, replacing variables where appropriate.
The ``toughest'' part of this program is determining whether or not you have a good Sudoku or not.  One suggestion would be to keep track of a variable the counts the number of things that are invalid and if that variable's value is 0 at the end of the program, you have a valid Sudoku.

\vspace{-0.05in}
\subsection*{Additional Program Requirements}
\vspace{-0.05in}
\begin{itemize}
\item Make sure your program prints your name, the lab number, and the date. 
\item Make sure your program contains the comment header with the honor pledge, your name, lab number, date, and the purpose of the program. 
\item Make sure your program is documented properly, with the comments throughout your program whenever appropriate. 
\item Make sure your output
is neat (no missing spaces, no typos, etc.) and that your program is neat (indenting, etc.).
\item \textbf{You may not use array structures or loops for this assignment.}
\end{itemize}

\vspace{-0.2in}
\subsection*{Required Deliverables}
\vspace{-0.05in}

For this assignment you are invited to submit electronic versions of the following deliverables through both the Bitbucket
repository and in a signed and printed format.

\vspace{-0.05in}
\begin{enumerate}
  \item A completed, properly commented and formatted {\tt Lab8.java} and {\tt Lab8Main.java} programs.

  \item An output document containing an output obtained after running the {\tt Lab8Main} program.
        
\end{enumerate}
\vspace{-0.05in}

\noindent As you complete this step, you should make sure that you created a {\tt lab8/} directory within the Git
repository.  Then, you can save all of the required deliverables in the {\tt lab8/} directory---please see the course
instructor or a teaching assistant if you are not able to create your directory properly. 

\noindent In addition to turning in signed and printed copies of your code and output, share your program and the output
file with me through your Git repository by correctly using ``{\tt git add}'', ``{\tt git commit}'', and ``{\tt git
  push}'' commands. When you are done, please ensure that the Bitbucket Web site has a {\tt lab8/} directory in your
repository with the two files called {\tt Lab8.java, Lab8Main.java} and {\tt output}. Please see the course instructor
if you have questions about assignment submission.

% In addition to turning in signed and printed copies of your code and output, please share your program and the output
% file with me through your Git repository by correctly using the ``{\tt git add}'', ``{\tt git commit}'', and ``{\tt git
% push}'' commands. When you are done, please ensure that the Bitbucket Web site has a {\tt lab8/} directory in your
% repository with the four files called {\tt Lab8.java}, {\tt Lab8Main.java}, and {\tt output}. You
% should see the instructor or a teaching assistant if you have questions about submitting this assignment.

In adherence to the Honor Code, students should complete this assignment on an individual basis. While it is appropriate
for students in this class to have high-level conversations about the assignment, it is necessary to distinguish
carefully between the student who discusses the principles underlying a problem with others and the student who produces
assignments that are identical to, or merely variations on, someone else's work.  Deliverables that are nearly identical
to the work of others will be taken as evidence of violating the \mbox{Honor Code}.  

\end{document}

\begin{verbatim}
Sample Run

Welcome to the Sudoku Checker v1.0!

This program checks simple, small, 4x4 Sudoku grids for correctness. Each column, row and 2x2 region contains the numbers 1 through 4 only once.

To check your Sudoku, enter your board one row at a time, with each digit separated by a space.  Hit ENTER at the end of a row.

Enter Row 1: 1 2 3 4
Enter Row 2: 3 4 2 1
Enter Row 3: 1 2 3 4
Enter Row 4: 4 3 2 1

Thank you.  Now checking ...

REG-1:GOOD
REG-2:GOOD
REG-3:GOOD
REG-4:GOOD
ROW-1:GOOD
ROW-2:GOOD
ROW-3:GOOD
ROW-4:GOOD
COL-1:BAD
COL-2:BAD
COL-3:GOOD
COL-4:GOOD

SUDO:INVALID



Sample Run
Welcome to the Sudoku Checker v1.0!

This program checks simple, small, 4x4 Sudoku grids for correctness. Each column, row and 2x2 region contains the numbers 1 through 4 only once.

To check your Sudoku, enter your board one row at a time, with each digit separated by a space.  Hit ENTER at the end of a row.

Enter Row 1: 3 2 4 1
Enter Row 2: 4 1 3 2
Enter Row 3: 1 4 2 3
Enter Row 4: 2 3 1 4

REG-1:GOOD
REG-2:GOOD
REG-3:GOOD
REG-4:GOOD
ROW-1:GOOD
ROW-2:GOOD
ROW-3:GOOD
ROW-4:GOOD
COL-1:GOOD
COL-2:GOOD
COL-3:GOOD
COL-4:GOOD

SUDO:VALID

\end{verbatim}



\end{document}
