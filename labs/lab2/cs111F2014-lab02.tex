%!TEX root=cs111F2014-lab02.tex
% mainfile: cs111F2014-lab02.tex

%!TEX root=cs111F2014-lab09.tex
% mainfile: cs111F2014-lab09.tex

% CS 111 style
% Typical usage (all UPPERCASE items are optional):
%       \input 111pre
%       \begin{document}
%       \MYTITLE{Title of document, e.g., Lab 1\\Due ...}
%       \MYHEADERS{short title}{other running head, e.g., due date}
%       \PURPOSE{Description of purpose}
%       \SUMMARY{Very short overview of assignment}
%       \DETAILS{Detailed description}
%         \SUBHEAD{if needed} ...
%         \SUBHEAD{if needed} ...
%          ...
%       \HANDIN{What to hand in and how}
%       \begin{checklist}
%       \item ...
%       \end{checklist}
% There is no need to include a "\documentstyle."
% However, there should be an "\end{document}."
%
%===========================================================
\documentclass[11pt,twoside,titlepage]{article}
%%NEED TO ADD epsf!!
\usepackage{threeparttop}
\usepackage{graphicx}
\usepackage{latexsym}
\usepackage{color}
\usepackage{listings}
\usepackage{fancyvrb}
%\usepackage{pgf,pgfarrows,pgfnodes,pgfautomata,pgfheaps,pgfshade}
\usepackage{tikz}
\usepackage[normalem]{ulem}
\tikzset{
    %Define standard arrow tip
%    >=stealth',
    %Define style for boxes
    oval/.style={
           rectangle,
           rounded corners,
           draw=black, very thick,
           text width=6.5em,
           minimum height=2em,
           text centered},
    % Define arrow style
    arr/.style={
           ->,
           thick,
           shorten <=2pt,
           shorten >=2pt,}
}
\usepackage[noend]{algorithmic}
\usepackage[noend]{algorithm}
\newcommand{\bfor}{{\bf for\ }}
\newcommand{\bthen}{{\bf then\ }}
\newcommand{\bwhile}{{\bf while\ }}
\newcommand{\btrue}{{\bf true\ }}
\newcommand{\bfalse}{{\bf false\ }}
\newcommand{\bto}{{\bf to\ }}
\newcommand{\bdo}{{\bf do\ }}
\newcommand{\bif}{{\bf if\ }}
\newcommand{\belse}{{\bf else\ }}
\newcommand{\band}{{\bf and\ }}
\newcommand{\breturn}{{\bf return\ }}
\newcommand{\mod}{{\rm mod}}
\renewcommand{\algorithmiccomment}[1]{$\rhd$ #1}
\newenvironment{checklist}{\par\noindent\hspace{-.25in}{\bf Checklist:}\renewcommand{\labelitemi}{$\Box$}%
\begin{itemize}}{\end{itemize}}
\pagestyle{threepartheadings}
\usepackage{url}
\usepackage{wrapfig}
\usepackage{hyperref}
%=========================
% One-inch margins everywhere
%=========================
\setlength{\topmargin}{0in}
\setlength{\textheight}{8.5in}
\setlength{\oddsidemargin}{0in}
\setlength{\evensidemargin}{0in}
\setlength{\textwidth}{6.5in}
%===============================
%===============================
% Macro for document title:
%===============================
\newcommand{\MYTITLE}[1]%
   {\begin{center}
     \begin{center}
     \bf
     CMPSC 111\\Introduction to Computer Science I\\
     Fall 2014\\
     %Janyl Jumadinova\\
     %\url{http://cs.allegheny.edu/~jjumadinova/111}
     \medskip
     \end{center}
     \bf
     #1
     \end{center}
}
%================================
% Macro for headings:
%================================
\newcommand{\MYHEADERS}[2]%
   {\lhead{#1}
    \rhead{#2}
    \immediate\write16{}
    \immediate\write16{DATE OF HANDOUT?}
    \read16 to \dateofhandout
    \lfoot{\sc Handed out on \dateofhandout}
    \immediate\write16{}
    \immediate\write16{HANDOUT NUMBER?}
    \read16 to\handoutnum
    \rfoot{Handout \handoutnum}
   }

%================================
% Macro for bold italic:
%================================
\newcommand{\bit}[1]{{\textit{\textbf{#1}}}}

%=========================
% Non-zero paragraph skips.
%=========================
\setlength{\parskip}{1ex}

%=========================
% Create various environments:
%=========================
\newcommand{\PURPOSE}{\par\noindent\hspace{-.25in}{\bf Purpose:\ }}
\newcommand{\SUMMARY}{\par\noindent\hspace{-.25in}{\bf Summary:\ }}
\newcommand{\DETAILS}{\par\noindent\hspace{-.25in}{\bf Details:\ }}
\newcommand{\HANDIN}{\par\noindent\hspace{-.25in}{\bf Hand in:\ }}
\newcommand{\SUBHEAD}[1]{\bigskip\par\noindent\hspace{-.1in}{\sc #1}\\}
%\newenvironment{CHECKLIST}{\begin{itemize}}{\end{itemize}}

\begin{document}
\MYTITLE{Lab 2\\11 September 2014\\Due Thursday, 18 September by 2:30 pm}
%\MYHEADERS{Lab 2}{Due 12 Sep, 2pm}

\subsection*{Objectives}

To develop a template for a Java program to use during this and future labs; learn standard ways of organizing and
preparing the lab; write a program to perform a simple calculation.

\subsection*{General Guidelines for Labs}

\begin{itemize}
\item
{\bf Work on the Alden Hall computers.} If you want to work on a different
machine, be sure to transfer your programs to the Alden
machines and re-run them before submitting.
\item
{\bf Update your repository often!} You should add, commit 
and push your updated files each time you work on them.  I will not grade 
your programs until the due date has passed.
\item
{\bf Review the Honor Code policy.} You
may discuss programs with others, but programs that are nearly identical
to others will be taken as evidence of violating the Honor Code.
\end{itemize}

\subsection*{Reading Assignment}

Please review the handout on ``Tips on Using Linux and Gvim'' (available in the shared repository in the ``handouts''
  directory).  Also review lecture slides and sections 2.2--2.6 in your textbook.

\subsection*{Create a Program Template}

Every program you write will have header comments, a ``{\tt main}'' method, some statements that print your name and the
date, etc. Rather than typing this in every time you have to write a Java program, you are going to create one
``template file'' that can be copied into each of your future laboratory files. As we learn more Java, you will be
making changes to this file to accommodate new features. You will store this template in your directory/repository,
named {\tt cs111F2014-<your user name>}, that you created during practical 1, making a copy of it for each laboratory
assignment.

\begin{sloppypar}
Go to the {\tt cs111F2014-<your user name>} directory and type the command ``{\tt gvim Template.java}''.  Create a Java
program template that you can fill in for each laboratory assignment.  See a program below for an example of what your
template should look like.  Please note that this program will not compile and cannot be run.  You only need to create
this template file once, thus allowing you to re-use it in the future laboratory sessions.
\end{sloppypar}

\newpage
\begin{Verbatim}[commandchars=\\\{\}]
     //*************************************
     // Honor Code: The work I am submitting is a result of my own thinking and efforts.
     // Your Name \textcolor{red}{\rm\em [Replace with your name]}
     // CMPSC 111 Fall 2014
     // Lab # \textcolor{red}{\rm\em [When you copy this file, fill in the lab number]}
     // Date: mmm dd yyyy \textcolor{red}{\rm\em [When you copy this file, fill in the date]}
     //
     // Purpose: ... \textcolor{red}{\rm\em [When you copy this file, describe the program]}
     //*************************************     
     import java.util.Date; // needed for printing today's date
     
     public class Xxxxxx \textcolor{red}{\rm\em [When you copy this file, replace with actual file name]}
     \{
         //----------------------------
         // main method: program execution begins here
         //----------------------------
         public static void main(String[] args)
         \{
            // Label output with name and date:
            System.out.println("Your Name\textbackslash{}nLab #\textbackslash{}n" + new Date() + "\textbackslash{}n");
     
            // Variable dictionary:
            \textcolor{red}{\rm\em [Declare variables and use comments to explain their meanings]}    
         \}
     \}
\end{Verbatim}

\noindent Save this file using the ``File/Save'' menu or by typing ``{\tt :w}'' when you are not in insert mode. At this
point, your {\tt cs111F2014-<your user name>} directory should contain a file named ``{\tt Template.java}.'' You may now
close {\tt gvim} editor. Make sure that you store your file in your Bitbucket repository!

\subsection*{Create a New Directory}

In your {\tt cs111F2014-<your user name>} directory type the command ``{\tt mkdir lab2}'' to create a new directory for lab 2.
\noindent Type ``{\tt cd lab2}'' to move to the new directory.  

\subsection*{Write a Java Program to Perform a Simple Calculation}
You are going to write a program that performs some kind of simple unit
conversion. I leave the specific problem up to you; I just ask that it be 
``G-rated'' and in good taste.

\noindent{\bf Creating the File:} Type ``{\tt gvim Lab2.java}''. Inside the
{\tt gvim} editor, make sure you are {\em not} in insert mode. (The word
``{\tt --INSERT--}'' should {\em not} appear in the lower left corner.) Type the
following exactly as shown; it will appear in the bottom line of the {\tt gvim}
window:
\begin{center}
\verb$:r  ../Template.java$
\end{center}
This ``reads'' your {\tt Template.java} file into your new {\tt Lab2} program.
You only need to do this once!  Now you need to edit {\tt Lab2.java} program with your name, the lab number, date, etc.

\noindent In your program for this laboratory assignment you will need to create variables and assign values to them. To create a variable you must declare it by specifying its type (String, integer, etc.) and the name of the variable that you chose (for example, {\tt int count;}). 
To assign the value to a variable after you have declared it, you need to write an assignment statement using an assignment operator `=', as, for example, {\tt count  = 0;} 
You may combine the variable declaration and assignment into one statement as: 
{\tt int count = 0;} 
You may print variables by incorporating them into a print or println statement by using `+' and the name of the variable as: {\tt System.out.println("My first variable is "+count);} 

 Complete the laboratory assignment by writing the Java statements needed to perform several simple
arithmetic calculations and print the results. See an example below. Obviously you may not use this!

\begin{verbatim}
     /* Honor Code: The work I am submitting is a result of my own thinking and efforts.
        Janyl Jumadinova
        CMPSC 111 Fall 2014
        Lab 2
        Date: September 11, 2014
     
        Purpose: to compute and print the number of yards between the Earth and the moon, 
        then print lunar maximum and minimum temperatures in both celsius and fahrenheit. */
     import java.util.Date; // needed for printing today's date
     
     public class MoonDistance
     {
         // main method: program execution begins here
         public static void main(String[] args)
         {
            // Label output with name and date:
            System.out.println("Your Name\nLab 2\n" + new Date() + "\n");     
            // Variables:
            int milesToMoon = 238900;   // distance to moon in miles
            int ydsPerMile = 1760;      // number of yards in a mile
            int ydsToMoon;              // number of yards to the moon           
            // Compute values:
            ydsToMoon = milesToMoon * ydsPerMile;
            
            System.out.println("Distance to the moon in miles: " + milesToMoon);
            System.out.println("The number of yards per mile: " +ydsPerMile);
            System.out.println("The number of yards from the earth");
            System.out.println("to the moon is " + ydsToMoon);
         }
     }
\end{verbatim}

\noindent \textbf{Note this example shows only one calculation using the multiplication operator '*'. Your program needs to use at least three different arithmetic operators. }

\subsubsection*{Program requirements}
\noindent Your program must:
\begin{itemize}
\item
Have a  comments header section, containing the Honor code, your name, lab number and the purpose statement for the lab. These items will not be printed since they are comments.
\item
Print your name, the lab number, and the current date and time (using
``{\tt new Date()}''---imitate the lab 1 program).
\item
Declare and use at least three variables (try using variables of different types).
\item
Some of your variables should be initialized with constant values; others
should be assigned the results of calculations (see example below). 
\item
Use at least three of the five arithmetic operators \verb$+$, \verb$-$, \verb$*$,
\verb$/$, or \verb$%$
\item
Once their values are known, print the values of all variables, labeled appropriately. 
\end{itemize}


\noindent In addition, for full credit you must:
\begin{itemize}
\item
Make sure the output printed by your program has a
pleasing appearance---for instance, you should have spaces between words;
lines should not be longer than the width of the screen, forcing them to ``wrap''
to the next line. (You do not need to worry about the way
fractional values are displayed.)
\item
Make sure your program is properly indented.
\item
Make sure you have inserted comments describing the program's purpose and
comments describing what each of the variables represents. Use both comment styles you have seen in class ($//$ and $/* ...   */$).
\item
Use variable names that ``make sense''
\end{itemize}


\subsection*{The Compile/Execute Cycle}
In your terminal window, still in the {\tt lab2} directory, type:
\begin{quote}
{\tt javac Lab2.java} 
\end{quote}
If there are errors, try to figure them out and correct them. Ask for help if
you don't understand the error messages. Be sure to watch out for
uppercase/lowercase errors, missing semicolons, etc.

When you have corrected the errors, type:
\begin{quote}
{\tt java Lab2}
\end{quote}
Did you get the desired result? If not, repair the program and go back to
the {\tt javac} command to re-compile it.

The cycle goes on like that: {\tt javac} is used to re-compile the
program every time you make a change
to the file. Use {\tt java} to execute the program once {\tt javac}
finds no more errors.

\subsection*{Required Deliverables}
This assignment asks you to submit electronic versions of the following deliverables through
a bitbucket repository that you created during practical 1: 
\begin{itemize}
	\item A completed, properly commented and formatted {\tt Lab2.java} program.
	\item The output from running {\tt Lab2} in the terminal window. You may use {\tt gvim} to save
	your output as follows: using the mouse,
select everything from the ``{\tt java Lab2}'' command to the end of your output.
Right-click on the selected text and copy it.
Type ``{\tt gvim output}'' [Note that this {\em not} a Java program!]
and use the ``paste'' command to paste your program output into the file.
 Save this file.
\end{itemize}

Share your program and the output file through your repository with me by using `add', `commit' and `push' commands correctly. When you are done, please check through bitbucket website that your `lab2' directory in your repository contains two files: {\tt Lab2.java} and {\tt output}.

\end{document}
