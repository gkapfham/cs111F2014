%!TEX root=cs111F2014-lab10.tex
% mainfile: cs111F2014-lab10.tex

%!TEX root=cs111F2014-lab09.tex
% mainfile: cs111F2014-lab09.tex

% CS 111 style
% Typical usage (all UPPERCASE items are optional):
%       \input 111pre
%       \begin{document}
%       \MYTITLE{Title of document, e.g., Lab 1\\Due ...}
%       \MYHEADERS{short title}{other running head, e.g., due date}
%       \PURPOSE{Description of purpose}
%       \SUMMARY{Very short overview of assignment}
%       \DETAILS{Detailed description}
%         \SUBHEAD{if needed} ...
%         \SUBHEAD{if needed} ...
%          ...
%       \HANDIN{What to hand in and how}
%       \begin{checklist}
%       \item ...
%       \end{checklist}
% There is no need to include a "\documentstyle."
% However, there should be an "\end{document}."
%
%===========================================================
\documentclass[11pt,twoside,titlepage]{article}
%%NEED TO ADD epsf!!
\usepackage{threeparttop}
\usepackage{graphicx}
\usepackage{latexsym}
\usepackage{color}
\usepackage{listings}
\usepackage{fancyvrb}
%\usepackage{pgf,pgfarrows,pgfnodes,pgfautomata,pgfheaps,pgfshade}
\usepackage{tikz}
\usepackage[normalem]{ulem}
\tikzset{
    %Define standard arrow tip
%    >=stealth',
    %Define style for boxes
    oval/.style={
           rectangle,
           rounded corners,
           draw=black, very thick,
           text width=6.5em,
           minimum height=2em,
           text centered},
    % Define arrow style
    arr/.style={
           ->,
           thick,
           shorten <=2pt,
           shorten >=2pt,}
}
\usepackage[noend]{algorithmic}
\usepackage[noend]{algorithm}
\newcommand{\bfor}{{\bf for\ }}
\newcommand{\bthen}{{\bf then\ }}
\newcommand{\bwhile}{{\bf while\ }}
\newcommand{\btrue}{{\bf true\ }}
\newcommand{\bfalse}{{\bf false\ }}
\newcommand{\bto}{{\bf to\ }}
\newcommand{\bdo}{{\bf do\ }}
\newcommand{\bif}{{\bf if\ }}
\newcommand{\belse}{{\bf else\ }}
\newcommand{\band}{{\bf and\ }}
\newcommand{\breturn}{{\bf return\ }}
\newcommand{\mod}{{\rm mod}}
\renewcommand{\algorithmiccomment}[1]{$\rhd$ #1}
\newenvironment{checklist}{\par\noindent\hspace{-.25in}{\bf Checklist:}\renewcommand{\labelitemi}{$\Box$}%
\begin{itemize}}{\end{itemize}}
\pagestyle{threepartheadings}
\usepackage{url}
\usepackage{wrapfig}
\usepackage{hyperref}
%=========================
% One-inch margins everywhere
%=========================
\setlength{\topmargin}{0in}
\setlength{\textheight}{8.5in}
\setlength{\oddsidemargin}{0in}
\setlength{\evensidemargin}{0in}
\setlength{\textwidth}{6.5in}
%===============================
%===============================
% Macro for document title:
%===============================
\newcommand{\MYTITLE}[1]%
   {\begin{center}
     \begin{center}
     \bf
     CMPSC 111\\Introduction to Computer Science I\\
     Fall 2014\\
     %Janyl Jumadinova\\
     %\url{http://cs.allegheny.edu/~jjumadinova/111}
     \medskip
     \end{center}
     \bf
     #1
     \end{center}
}
%================================
% Macro for headings:
%================================
\newcommand{\MYHEADERS}[2]%
   {\lhead{#1}
    \rhead{#2}
    \immediate\write16{}
    \immediate\write16{DATE OF HANDOUT?}
    \read16 to \dateofhandout
    \lfoot{\sc Handed out on \dateofhandout}
    \immediate\write16{}
    \immediate\write16{HANDOUT NUMBER?}
    \read16 to\handoutnum
    \rfoot{Handout \handoutnum}
   }

%================================
% Macro for bold italic:
%================================
\newcommand{\bit}[1]{{\textit{\textbf{#1}}}}

%=========================
% Non-zero paragraph skips.
%=========================
\setlength{\parskip}{1ex}

%=========================
% Create various environments:
%=========================
\newcommand{\PURPOSE}{\par\noindent\hspace{-.25in}{\bf Purpose:\ }}
\newcommand{\SUMMARY}{\par\noindent\hspace{-.25in}{\bf Summary:\ }}
\newcommand{\DETAILS}{\par\noindent\hspace{-.25in}{\bf Details:\ }}
\newcommand{\HANDIN}{\par\noindent\hspace{-.25in}{\bf Hand in:\ }}
\newcommand{\SUBHEAD}[1]{\bigskip\par\noindent\hspace{-.1in}{\sc #1}\\}
%\newenvironment{CHECKLIST}{\begin{itemize}}{\end{itemize}}

\begin{document}
\MYTITLE{Lab 10 for Sections 03 and 04 \\ 13 November 2014\\
Due Thursday, 20 November by 2:30pm }

\vspace{-0.2in}
\subsection*{Objectives}
\vspace{-0.05in}

To learn more about how to use iteration constructs, such as the {\tt for} loop, when writing Java programs. In
addition, to explore some of the advanced graphics features that Java provides.  Finally, to learn how to conduct an
empirical study, using operating system timers, to evaluate how the number of {\tt for} loop iterations influences the
execution time of a Java program.

\vspace{-0.15in}
\subsection*{General Guidelines for Labs}
\vspace{-0.05in}

\begin{itemize}
\item
{\bf Work on the Alden Hall computers.} If you want to work on a different
machine, be sure to transfer your programs to the Alden
machines and re-run them before submitting.
\item
  {\bf Update your repository often!} You should {\tt add}, {\tt commit}, 
  and {\tt push} your updated files each time you work on them.  I will not grade 
your programs until the due date has passed.
\item
{\bf Review the Honor Code policy.} You
may discuss programs with others, but programs that are nearly identical
to others' will be taken as evidence of violating the Honor Code.
\end{itemize}

\vspace{-0.25in}
\subsection*{Reading Assignment}
\vspace{-0.05in}

Review Section 6.4 to learn more about the structure and behavior of the {\tt for} loop provided by the Java programming
language, paying close attention to the three key parts of the {\tt for} loop's declaration. To learn more about
fractals, and the Mandelbrot set that we will visualize in this laboratory assignment, please study the following Web
site available at \url{http://jonisalonen.com/2013/lets-draw-the-mandelbrot-set/}. Students who have never explored the
concept of fractals are also encouraged to review \url{http://en.wikipedia.org/wiki/Mandelbrot_set}.

\vspace{-0.1in}
\subsection*{Create a New Directory and Starting the Project}
\vspace{-0.05in}

After changing into the ``{\tt cs111F2014-share/}'' directory, which contains our course's version control repository,
you should type the command ``{\tt git pull}'' to download the source code for this laboratory assignment.  In your own
``{\tt cs111F2014-<your user name>}'' repository inside the ``{\tt labs/}'' directory, create a directory called ``{\tt
  lab10}''. Using the method described in a previous laboratory assignment, please copy the source code from the share
repository to your own repository. Now, change into the ``{\tt labs/lab10/}'' directory, in your own Git repository, and
use ``{\tt gvim}'' to study the source code of the provided files. What methods do these classes provide? How do they
work? While you do not need to understand the details of the provided source code, you should be able to add explanatory
comments that highlight the basic points about how this program works. Students who do not understand these two programs
should ask the course instructor for assistance.


\vspace{-0.1in}
\subsection*{Understanding the Todo List Manager}
\vspace{-0.05in}










\vspace{-0.3in}
\subsection*{Required Deliverables}
\vspace{-0.05in}

For this assignment you are invited to submit versions of the following deliverables through both the Bitbucket
repository and in a signed and printed format.

\vspace{-0.1in}
\begin{enumerate}
    \setlength{\itemsep}{0pt}

  \item Completed, fully commented, and properly formatted versions of the three source code files.
  \item An output file, called {\tt output}, containing outputs from five runs of {\tt TodoListMain}.
  \item The final version of the {\tt todo.txt} file that you used to both test and complete this lab.
  \item A written reflection, saved in {\tt reflection}, about the challenges you faced during this lab.
        
\end{enumerate}
\vspace{-0.1in}

In addition to turning in signed and printed copies of your code and output, share your source code and other required
files with me through your Git repository by correctly using ``{\tt git add}'', ``{\tt git commit}'', and ``{\tt git
  push}'' commands. When you are done, please ensure that the Bitbucket Web site has a ``{\tt lab9/}'' directory in your
repository with the three Java files in the list of deliverables and the other files. Please see the instructor if
you have questions about assignment submission.

In adherence to the Honor Code, students should complete this assignment on an individual basis. While it is appropriate
for students in this class to have high-level conversations about the assignment, it is necessary to distinguish
carefully between the student who discusses the principles underlying a problem with others and the student who produces
assignments that are identical to, or merely variations on, someone else's work.  Deliverables that are nearly identical
to the work of others will be taken as evidence of violating Allegheny College's \mbox{Honor Code}.  

\end{document}


