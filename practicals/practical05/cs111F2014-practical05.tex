%!TEX root=cs111F2014-practical05.tex
% mainfile: cs111F2014-practical05.tex

%!TEX root=cs111F2014-lab09.tex
% mainfile: cs111F2014-lab09.tex

% CS 111 style
% Typical usage (all UPPERCASE items are optional):
%       \input 111pre
%       \begin{document}
%       \MYTITLE{Title of document, e.g., Lab 1\\Due ...}
%       \MYHEADERS{short title}{other running head, e.g., due date}
%       \PURPOSE{Description of purpose}
%       \SUMMARY{Very short overview of assignment}
%       \DETAILS{Detailed description}
%         \SUBHEAD{if needed} ...
%         \SUBHEAD{if needed} ...
%          ...
%       \HANDIN{What to hand in and how}
%       \begin{checklist}
%       \item ...
%       \end{checklist}
% There is no need to include a "\documentstyle."
% However, there should be an "\end{document}."
%
%===========================================================
\documentclass[11pt,twoside,titlepage]{article}
%%NEED TO ADD epsf!!
\usepackage{threeparttop}
\usepackage{graphicx}
\usepackage{latexsym}
\usepackage{color}
\usepackage{listings}
\usepackage{fancyvrb}
%\usepackage{pgf,pgfarrows,pgfnodes,pgfautomata,pgfheaps,pgfshade}
\usepackage{tikz}
\usepackage[normalem]{ulem}
\tikzset{
    %Define standard arrow tip
%    >=stealth',
    %Define style for boxes
    oval/.style={
           rectangle,
           rounded corners,
           draw=black, very thick,
           text width=6.5em,
           minimum height=2em,
           text centered},
    % Define arrow style
    arr/.style={
           ->,
           thick,
           shorten <=2pt,
           shorten >=2pt,}
}
\usepackage[noend]{algorithmic}
\usepackage[noend]{algorithm}
\newcommand{\bfor}{{\bf for\ }}
\newcommand{\bthen}{{\bf then\ }}
\newcommand{\bwhile}{{\bf while\ }}
\newcommand{\btrue}{{\bf true\ }}
\newcommand{\bfalse}{{\bf false\ }}
\newcommand{\bto}{{\bf to\ }}
\newcommand{\bdo}{{\bf do\ }}
\newcommand{\bif}{{\bf if\ }}
\newcommand{\belse}{{\bf else\ }}
\newcommand{\band}{{\bf and\ }}
\newcommand{\breturn}{{\bf return\ }}
\newcommand{\mod}{{\rm mod}}
\renewcommand{\algorithmiccomment}[1]{$\rhd$ #1}
\newenvironment{checklist}{\par\noindent\hspace{-.25in}{\bf Checklist:}\renewcommand{\labelitemi}{$\Box$}%
\begin{itemize}}{\end{itemize}}
\pagestyle{threepartheadings}
\usepackage{url}
\usepackage{wrapfig}
\usepackage{hyperref}
%=========================
% One-inch margins everywhere
%=========================
\setlength{\topmargin}{0in}
\setlength{\textheight}{8.5in}
\setlength{\oddsidemargin}{0in}
\setlength{\evensidemargin}{0in}
\setlength{\textwidth}{6.5in}
%===============================
%===============================
% Macro for document title:
%===============================
\newcommand{\MYTITLE}[1]%
   {\begin{center}
     \begin{center}
     \bf
     CMPSC 111\\Introduction to Computer Science I\\
     Fall 2014\\
     %Janyl Jumadinova\\
     %\url{http://cs.allegheny.edu/~jjumadinova/111}
     \medskip
     \end{center}
     \bf
     #1
     \end{center}
}
%================================
% Macro for headings:
%================================
\newcommand{\MYHEADERS}[2]%
   {\lhead{#1}
    \rhead{#2}
    \immediate\write16{}
    \immediate\write16{DATE OF HANDOUT?}
    \read16 to \dateofhandout
    \lfoot{\sc Handed out on \dateofhandout}
    \immediate\write16{}
    \immediate\write16{HANDOUT NUMBER?}
    \read16 to\handoutnum
    \rfoot{Handout \handoutnum}
   }

%================================
% Macro for bold italic:
%================================
\newcommand{\bit}[1]{{\textit{\textbf{#1}}}}

%=========================
% Non-zero paragraph skips.
%=========================
\setlength{\parskip}{1ex}

%=========================
% Create various environments:
%=========================
\newcommand{\PURPOSE}{\par\noindent\hspace{-.25in}{\bf Purpose:\ }}
\newcommand{\SUMMARY}{\par\noindent\hspace{-.25in}{\bf Summary:\ }}
\newcommand{\DETAILS}{\par\noindent\hspace{-.25in}{\bf Details:\ }}
\newcommand{\HANDIN}{\par\noindent\hspace{-.25in}{\bf Hand in:\ }}
\newcommand{\SUBHEAD}[1]{\bigskip\par\noindent\hspace{-.1in}{\sc #1}\\}
%\newenvironment{CHECKLIST}{\begin{itemize}}{\end{itemize}}

\begin{document}
\MYTITLE{Practical 5\\23--24 October 2014\\Due in Bitbucket by midnight of the day of your practical \\ ``Checkmark'' grade}

%\vspace*{-.2in}
\subsection*{Summary}
\vspace*{-.05in}

As a means of practicing the extension of your own classes and methods and better understanding the structure of
different classes, you will study the given Java programs and modify them by adding more functionality to the given
methods.  Additionally, you will use the Lightweight Java Visualizer (LJV) to automatically create diagrams, like those
seen in Chapter 4, that depict the state of Java objects.  Then, using the ``{\tt git add}'', ``{\tt git commit}'', and
``{\tt git push}'' commands you will upload your modified programs, object visualizations, and the output you obtain
from running your programs to your Git repository hosted by Bitbucket.  

\vspace*{-.15in}
\subsection*{Review the Textbook}
\vspace*{-.05in}
Be sure to read Sections 4.1 through 4.3 of your book to learn more about writing your own classes, constructors and methods.  As you review this material, try to make a list of questions about concepts that you do not yet fully
understand.  In addition to discussing these questions with the teaching assistants and the course instructor, try to
answer them as you complete this assignment.

\vspace*{-.15in}
\subsection*{Save the Example Programs}
\vspace*{-.05in}
Get the files ``{\tt Octopus.java}'', ``{\tt Utensil.java},'' and
``{\tt Practical5.java}'' from the course repository. Copy these files 
into your own Bitbucket repository into a directory called {\tt /practical05}. Study 
these programs first and make sure you understand them. To compile them, type
{\tt javac *.java} and to run them, type {\tt java Practical5}.

\vspace*{-.15in}
\subsection*{Modify the Programs} 
\vspace*{-.05in}
\begin{enumerate}
\item 
\begin{itemize}
\item 
Edit the file {\tt Octopus.java} and find the constructor in this class. In the constructor,
there is one parameter {\tt n}, which contains a {\tt String}. 
Change this by adding one more parameter, {\tt a}, of type {\tt int}.
This is the age of the octopus. Save this in the appropriate instance
variable (imitating what was done for the name). 
\item 
\noindent Edit the file {\tt Practical5.java} and look for the place where variable
{\tt ocky} is defined to be a {\tt new Octopus}. Add an ``age'' to this
so that we are specifying two things, not one, in the construction.
Delete or comment out the ``{\tt ocky.setAge(10)}'' method call in the next line. 
\item 
\noindent Recompile and re-run the program and see if it correctly provides the age
you specified.
\end{itemize} \newpage
\item 
\begin{itemize}
\item
Edit the file {\tt Practical5.java}. Declare a second {\tt Octopus} variable (don't
just change the name of the one that's there---create another one) and
assign it any name and age that you want. 
\item
\noindent Create a second {\tt Utensil} of any type you wish, imitating the declaration and
initialization of {\tt spat}.
Assign a cost and a color to this utensil. Assign this utensil to the new
Octopus you created. 
\item
\noindent Print out the name, age, weight, and favorite utensil of your new octopus.
Print out the type, cost, and color of your new utensil.
\end{itemize}
\end{enumerate}

\vspace*{-.15in}
\subsection*{Completing the Practical Assignment}
\vspace*{-.1in}
To finish this assignment and earn a ``checkmark'', you should 
submit the {\tt Octopus.java} and {\tt Practical5.java} files you just edited
in your Bitbucket repository by
using appropriate {\tt git} commands.

\vspace*{-.15in}
\subsection*{General Guidelines for Practical Sessions}
\vspace*{-.05in}
\begin{itemize}
\item {\bf Submit \textbf{\textit{Something}}.} Your grade for this assignment is a ``checkmark'' indicating whether you
  did or did not complete the work and submit something to the Bitbucket repository using the ``{\tt git add}'', ``{\tt
    git commit}'', and ``{\tt git push}'' commands.

\item {\bf Update Your Repository Often!} You should {\tt add}, {\tt commit}, and {\tt push} your updated files each
  time you work on them, always including descriptive messages about each code change.

\item {\bf Review the Honor Code Policy on the Syllabus.} Remember that while you may discuss your writing with other
  students in the course, text that is nearly identical to, or merely variations on, the work of others will be
  taken as evidence of violating the \mbox{Honor Code}.
\end{itemize}


\end{document}
