%!TEX root=cs111F2014-practical03.tex
% mainfile: cs111F2014-practical03.tex

\input{111pre}
\begin{document}
\MYTITLE{Practical 3\\18--19 September 2014\\Due in Bitbucket by midnight of the day of your practical \\ ``Checkmark'' grade}

\subsection*{Summary}

As a means of practicing with user input and output and the declaration of variables, and the creation of expressions in
the Java programming language, you will write a ``Mad Libs'' program involving words, numbers, and some calculations.
Then, using the {\tt git add}, {\tt git commit}, and {\tt git push} commands you should upload it to your Git repository
hosted by Bitbucket.  

\noindent
If you have never heard of Mad Libs, please visit this Web site: \url{http://www.madlibs.com/}. 

\subsection*{Review the Textbook}

Be sure to read Sections 3.1 through 3.6 of your book to learn more about variables, data types, expressions, and user
input.  As you review this material, try to make a list of questions about concepts that you do not yet fully
understand.  In addition to discussing these questions with the teaching assistants and the course instructor, try to
answer them as your complete this assignment.

\subsection*{Exercise: Implement a ``Mad Libs'' Program} 

Your program should ask the user to enter words and numbers, then print out a story using those words and numbers. The
numbers should be used to calculate something which will also be printed out. See Figure \ref{mad} for an example, but I
hope you will create your own story!

\begin{figure}[tb]
\begin{Verbatim}[commandchars=\\\{\}]
aldenv27:gkapfham$ \textcolor{red}{javac MadLib.java}
aldenv27:gkapfham$ \textcolor{red}{java MadLib}
Your name, Practical 3
Thu Sep 17 13:11:17 EST 2014

Enter a singular noun: \textcolor{red}{noggin}
Enter an adjective: \textcolor{red}{verboten}
Enter another adjective: \textcolor{red}{glitzy}
Enter a non-zero whole number: \textcolor{red}{32}
Enter another non-zero whole number: \textcolor{red}{42}
Enter any number: \textcolor{red}{5.43}
Enter a singular verb: \textcolor{red}{snooze}

----------------------------------

    Third-Grade Word Problem

If you own 32 verboten noggins,
and you purchase 42 glitzy noggins,
how many more noggins do you need to snooze 5.43 noggins?

Answer: You need -68.57 more noggins.
\end{Verbatim}
\caption{Sample ``Mad Libs'' output; user input is in red}
\label{mad}
\end{figure}

Remember to {\tt import java.util.Scanner} and to create a {\tt Scanner}
variable (named {\tt scan}, but you can name it something else if you want);
use things like \verb$... = scan.next()$, \verb$... = scan.nextInt()$, and
\verb$... = scan.nextDouble()$ to read in strings, integers, and general
numbers. Output should be neat and make good use of white space and
labeling.

At the end of the period, or by midnight of the day of your practical,
upload the file you just created (only the {\tt .java} file, nothing else).

If you were unable to complete the exercise and have nothing to upload,
please send me an email with the subject line ``Practical 4'' and
tell me what problems you encountered so that I can help you. (Actually,
email me if you had any problems or questions, even if you uploaded something.)


\subsection*{General Guidelines for Practical Sessions}
\begin{itemize}
\item
{\bf Experiment!} 
Practical sessions are for learning by doing
without the pressure of grades or ``right/wrong'' answers. So try
things!  The best way to learn is by trying things out.
\item
{\bf Submit \textbf{\textit{something}}.} Your grade is just 0 or 1,
depending on whether or not you attempt the work and upload something to 
Sakai. 
\item
{\bf Try to Finish During Class.} Practical exercises are not intended
to be the equal of laboratory assignments. If you are simply a slow
typist, I've given you until the end of the day, but ideally you should
upload a file, even a non-working one, by the end of the class period and 
be finished with it.
\item
{\bf Help One Another!} 
If your neighbor is struggling and you know what
to do, offer your help. Don't ``do the work'' for them, but advise them on what
to type or how to handle things.
\item
{\bf Review the Honor Code policy on the syllabus.} Remember that you
may discuss programs with others, but programs that are nearly identical
to others will be taken as evidence of violating the Honor Code.
\end{itemize}
\end{document}
