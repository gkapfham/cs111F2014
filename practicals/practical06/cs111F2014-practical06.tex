%!TEX root=cs111F2014-practical06.tex
% mainfile: cs111F2014-practical06.tex

%!TEX root=cs111F2014-lab09.tex
% mainfile: cs111F2014-lab09.tex

% CS 111 style
% Typical usage (all UPPERCASE items are optional):
%       \input 111pre
%       \begin{document}
%       \MYTITLE{Title of document, e.g., Lab 1\\Due ...}
%       \MYHEADERS{short title}{other running head, e.g., due date}
%       \PURPOSE{Description of purpose}
%       \SUMMARY{Very short overview of assignment}
%       \DETAILS{Detailed description}
%         \SUBHEAD{if needed} ...
%         \SUBHEAD{if needed} ...
%          ...
%       \HANDIN{What to hand in and how}
%       \begin{checklist}
%       \item ...
%       \end{checklist}
% There is no need to include a "\documentstyle."
% However, there should be an "\end{document}."
%
%===========================================================
\documentclass[11pt,twoside,titlepage]{article}
%%NEED TO ADD epsf!!
\usepackage{threeparttop}
\usepackage{graphicx}
\usepackage{latexsym}
\usepackage{color}
\usepackage{listings}
\usepackage{fancyvrb}
%\usepackage{pgf,pgfarrows,pgfnodes,pgfautomata,pgfheaps,pgfshade}
\usepackage{tikz}
\usepackage[normalem]{ulem}
\tikzset{
    %Define standard arrow tip
%    >=stealth',
    %Define style for boxes
    oval/.style={
           rectangle,
           rounded corners,
           draw=black, very thick,
           text width=6.5em,
           minimum height=2em,
           text centered},
    % Define arrow style
    arr/.style={
           ->,
           thick,
           shorten <=2pt,
           shorten >=2pt,}
}
\usepackage[noend]{algorithmic}
\usepackage[noend]{algorithm}
\newcommand{\bfor}{{\bf for\ }}
\newcommand{\bthen}{{\bf then\ }}
\newcommand{\bwhile}{{\bf while\ }}
\newcommand{\btrue}{{\bf true\ }}
\newcommand{\bfalse}{{\bf false\ }}
\newcommand{\bto}{{\bf to\ }}
\newcommand{\bdo}{{\bf do\ }}
\newcommand{\bif}{{\bf if\ }}
\newcommand{\belse}{{\bf else\ }}
\newcommand{\band}{{\bf and\ }}
\newcommand{\breturn}{{\bf return\ }}
\newcommand{\mod}{{\rm mod}}
\renewcommand{\algorithmiccomment}[1]{$\rhd$ #1}
\newenvironment{checklist}{\par\noindent\hspace{-.25in}{\bf Checklist:}\renewcommand{\labelitemi}{$\Box$}%
\begin{itemize}}{\end{itemize}}
\pagestyle{threepartheadings}
\usepackage{url}
\usepackage{wrapfig}
\usepackage{hyperref}
%=========================
% One-inch margins everywhere
%=========================
\setlength{\topmargin}{0in}
\setlength{\textheight}{8.5in}
\setlength{\oddsidemargin}{0in}
\setlength{\evensidemargin}{0in}
\setlength{\textwidth}{6.5in}
%===============================
%===============================
% Macro for document title:
%===============================
\newcommand{\MYTITLE}[1]%
   {\begin{center}
     \begin{center}
     \bf
     CMPSC 111\\Introduction to Computer Science I\\
     Fall 2014\\
     %Janyl Jumadinova\\
     %\url{http://cs.allegheny.edu/~jjumadinova/111}
     \medskip
     \end{center}
     \bf
     #1
     \end{center}
}
%================================
% Macro for headings:
%================================
\newcommand{\MYHEADERS}[2]%
   {\lhead{#1}
    \rhead{#2}
    \immediate\write16{}
    \immediate\write16{DATE OF HANDOUT?}
    \read16 to \dateofhandout
    \lfoot{\sc Handed out on \dateofhandout}
    \immediate\write16{}
    \immediate\write16{HANDOUT NUMBER?}
    \read16 to\handoutnum
    \rfoot{Handout \handoutnum}
   }

%================================
% Macro for bold italic:
%================================
\newcommand{\bit}[1]{{\textit{\textbf{#1}}}}

%=========================
% Non-zero paragraph skips.
%=========================
\setlength{\parskip}{1ex}

%=========================
% Create various environments:
%=========================
\newcommand{\PURPOSE}{\par\noindent\hspace{-.25in}{\bf Purpose:\ }}
\newcommand{\SUMMARY}{\par\noindent\hspace{-.25in}{\bf Summary:\ }}
\newcommand{\DETAILS}{\par\noindent\hspace{-.25in}{\bf Details:\ }}
\newcommand{\HANDIN}{\par\noindent\hspace{-.25in}{\bf Hand in:\ }}
\newcommand{\SUBHEAD}[1]{\bigskip\par\noindent\hspace{-.1in}{\sc #1}\\}
%\newenvironment{CHECKLIST}{\begin{itemize}}{\end{itemize}}

\begin{document}

\MYTITLE{Practical 6\\30--31 October 2014 \\ Due in Bitbucket by midnight of the day of your practical \\ ``Checkmark'' grade}

\subsection*{Summary}
\vspace*{-.05in}

In this practical you will create methods to determine if certain events occurred during the year given by the user. For
this task you will use {\tt if/else} statements and the boolean logic operators.  Finally, using the ``{\tt git add}'',
``{\tt git  commit}'', and ``{\tt git push}'' commands you will upload your modified programs and the output you obtain
from running your programs to your Git repository hosted by Bitbucket.  

\vspace*{-.1in}
\subsection*{Review the Textbook}
\vspace*{-.05in}

You may refer to sections 5.1--5.3 in your textbook to learn more about {\tt if/else} statements and boolean expressions. 

\vspace*{-.1in}
\subsection*{Save the Example Programs}
\vspace*{-.05in}
Get the files ``{\tt Practical6.java}'' and ``{\tt Practical6Main.java}'' from the shared course repository. Copy these files 
into your own Bitbucket repository into a directory called {\tt /practical06}. Study 
these programs first and make sure you understand them. Please note that these programs are not complete, and will not run correctly.

\vspace*{-.1in}
\subsection*{``Through All the Years''}
\vspace*{-.05in}
The quotation is from the College's Alma Mater. In the second class ({\tt Practical6.java}), you will write methods that for the user's input (from {\tt Practical6Main.java}), determine 
which of the following events occurs that year:
\begin{itemize}
\item
leap year
\item
emergence of the 17-year cicadas (more specifically, Brood II)
\item
peak year of sunspot activity
\end{itemize}

\noindent A year is a {\em leap year} if it is divisible by 4, {\em unless} it is a century
year. If it is a century year, it is a leap year if it is divisible by 400. For instance,
1968 and 1972 are leap years since they are divisible by 4; 1967 and 1970 are not.
The year 2000 is a leap year because it is divisible by 400; however, 1900 is not (even
though 1900 is divisible by 4---century years are treated differently).

\noindent The 17-year cicadas emerge from underground every 17 years. There are several 
``broods;'' the one we are interested in emerged in 2013. (So any year that differs
from 2013 by a multiple of 17 is also an emergence year, e.g., 2040, 1996, 1928, 3713.)

\noindent Sunspot activity usually peaks every 11 years. The year 2013 was supposed to be such
a ``solar max'' year. (So any year that differs from 2013 by a multiple of 11 should
also be a solar max year, e.g., 2002, 2024, 1793.)


\noindent The sample output is shown below:
\begin{Verbatim}[commandchars=\\\{\}]
      ewire23-29:practical6 jjumadinova$ \textcolor{red}{java Practical6Main}
            
      Enter a year between 1000 and 3000: \textcolor{red}{1452}
      1452 is a leap year
      It's an emergence year for Brood II of the 17-year cicadas
      It's a peak sunspot year.
      Thank you for using this program! 
\end{Verbatim}

\noindent Thoroughly test your program! Try the above date and some of your own as well.

\vspace*{-.1in}
\subsection*{Modify the Programs} 
\vspace*{-.05in}
\begin{itemize}
\item Edit the file {\tt Practical6.java} and find the constructor in this class. Make sure you understand what the constructor contains, when it gets called, and what happens after it is executed. Please ask the instructors or the teaching assistance if you are not clear about the purpose of the constructors. 
\item  In {\tt Practical6.java} and find three methods {\tt setLeapYear(), setCicadaYear(),} and {\tt  setSunspotYear()}. You task for this practical is to create the actions that these methods need to execute, as described in the previous section. 
\item Edit the file {\tt Practical6Main.java} by including method calls to {\tt setLeapYear(), setCicadaYear(),} and {\tt  setSunspotYear()}, or in other words, by writing Java statements that will invoke these three methods. You do not need to change anything else in this class.
\end{itemize}

\vspace*{-.15in}
\subsection*{Completing the Practical Assignment}
\vspace*{-.1in}
To finish this assignment and earn a ``checkmark'', you should 
submit the {\tt Practical6.java} and {\tt Practical6Main.java} files you just edited
in your Bitbucket repository by
using appropriate {\tt git} commands. You also need to submit an output file with several runs of your program.

\vspace*{-.15in}
\subsection*{General Guidelines for Practical Sessions}
\vspace*{-.05in}
\begin{itemize}
\item {\bf Submit \textbf{\textit{Something}}.} Your grade for this assignment is a ``checkmark'' indicating whether you
  did or did not complete the work and submit something to the Bitbucket repository using the ``{\tt git add}'', ``{\tt
    git commit}'', and ``{\tt git push}'' commands.

\item {\bf Update Your Repository Often!} You should {\tt add}, {\tt commit}, and {\tt push} your updated files each
  time you work on them, always including descriptive messages about each code change.

\item {\bf Review the Honor Code Policy on the Syllabus.} Remember that while you may discuss your writing with other
  students in the course, text that is nearly identical to, or merely variations on, the work of others will be
  taken as evidence of violating the \mbox{Honor Code}.
\end{itemize}


\end{document}
