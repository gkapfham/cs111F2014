%!TEX root=cs111F2014-lab09.tex
% mainfile: cs111F2014-lab09.tex

% CS 111 style
% Typical usage (all UPPERCASE items are optional):
%       \input 111pre
%       \begin{document}
%       \MYTITLE{Title of document, e.g., Lab 1\\Due ...}
%       \MYHEADERS{short title}{other running head, e.g., due date}
%       \PURPOSE{Description of purpose}
%       \SUMMARY{Very short overview of assignment}
%       \DETAILS{Detailed description}
%         \SUBHEAD{if needed} ...
%         \SUBHEAD{if needed} ...
%          ...
%       \HANDIN{What to hand in and how}
%       \begin{checklist}
%       \item ...
%       \end{checklist}
% There is no need to include a "\documentstyle."
% However, there should be an "\end{document}."
%
%===========================================================
\documentclass[11pt,twoside,titlepage]{article}
%%NEED TO ADD epsf!!
\usepackage{threeparttop}
\usepackage{graphicx}
\usepackage{latexsym}
\usepackage{color}
\usepackage{listings}
\usepackage{fancyvrb}
%\usepackage{pgf,pgfarrows,pgfnodes,pgfautomata,pgfheaps,pgfshade}
\usepackage{tikz}
\usepackage[normalem]{ulem}
\tikzset{
    %Define standard arrow tip
%    >=stealth',
    %Define style for boxes
    oval/.style={
           rectangle,
           rounded corners,
           draw=black, very thick,
           text width=6.5em,
           minimum height=2em,
           text centered},
    % Define arrow style
    arr/.style={
           ->,
           thick,
           shorten <=2pt,
           shorten >=2pt,}
}
\usepackage[noend]{algorithmic}
\usepackage[noend]{algorithm}
\newcommand{\bfor}{{\bf for\ }}
\newcommand{\bthen}{{\bf then\ }}
\newcommand{\bwhile}{{\bf while\ }}
\newcommand{\btrue}{{\bf true\ }}
\newcommand{\bfalse}{{\bf false\ }}
\newcommand{\bto}{{\bf to\ }}
\newcommand{\bdo}{{\bf do\ }}
\newcommand{\bif}{{\bf if\ }}
\newcommand{\belse}{{\bf else\ }}
\newcommand{\band}{{\bf and\ }}
\newcommand{\breturn}{{\bf return\ }}
\newcommand{\mod}{{\rm mod}}
\renewcommand{\algorithmiccomment}[1]{$\rhd$ #1}
\newenvironment{checklist}{\par\noindent\hspace{-.25in}{\bf Checklist:}\renewcommand{\labelitemi}{$\Box$}%
\begin{itemize}}{\end{itemize}}
\pagestyle{threepartheadings}
\usepackage{url}
\usepackage{wrapfig}
\usepackage{hyperref}
%=========================
% One-inch margins everywhere
%=========================
\setlength{\topmargin}{0in}
\setlength{\textheight}{8.5in}
\setlength{\oddsidemargin}{0in}
\setlength{\evensidemargin}{0in}
\setlength{\textwidth}{6.5in}
%===============================
%===============================
% Macro for document title:
%===============================
\newcommand{\MYTITLE}[1]%
   {\begin{center}
     \begin{center}
     \bf
     CMPSC 111\\Introduction to Computer Science I\\
     Fall 2014\\
     %Janyl Jumadinova\\
     %\url{http://cs.allegheny.edu/~jjumadinova/111}
     \medskip
     \end{center}
     \bf
     #1
     \end{center}
}
%================================
% Macro for headings:
%================================
\newcommand{\MYHEADERS}[2]%
   {\lhead{#1}
    \rhead{#2}
    \immediate\write16{}
    \immediate\write16{DATE OF HANDOUT?}
    \read16 to \dateofhandout
    \lfoot{\sc Handed out on \dateofhandout}
    \immediate\write16{}
    \immediate\write16{HANDOUT NUMBER?}
    \read16 to\handoutnum
    \rfoot{Handout \handoutnum}
   }

%================================
% Macro for bold italic:
%================================
\newcommand{\bit}[1]{{\textit{\textbf{#1}}}}

%=========================
% Non-zero paragraph skips.
%=========================
\setlength{\parskip}{1ex}

%=========================
% Create various environments:
%=========================
\newcommand{\PURPOSE}{\par\noindent\hspace{-.25in}{\bf Purpose:\ }}
\newcommand{\SUMMARY}{\par\noindent\hspace{-.25in}{\bf Summary:\ }}
\newcommand{\DETAILS}{\par\noindent\hspace{-.25in}{\bf Details:\ }}
\newcommand{\HANDIN}{\par\noindent\hspace{-.25in}{\bf Hand in:\ }}
\newcommand{\SUBHEAD}[1]{\bigskip\par\noindent\hspace{-.1in}{\sc #1}\\}
%\newenvironment{CHECKLIST}{\begin{itemize}}{\end{itemize}}

\begin{document}

\MYTITLE{Practical 8 \\ 20--21 November 2014 \\ Due in Bitbucket by midnight of the day of your practical \\ ``Checkmark'' grade}

\subsection*{Summary}
\vspace*{-.05in}

In this practical assignment, you will explore Java programs that create music using arrays and {\tt for} loops. In
addition, you will learn more about how to set the {\tt CLASSPATH} environment variable to load a third-party Java
archive that provides music playing and generating capabilities.

\vspace*{-.1in}
\subsection*{Review the Textbook}
\vspace*{-.05in}

To learn more about the concepts associated with arrays, please review the content in Section 8.2. Students who want to
learn more about the {\tt for} loop can investigate Section 6.4.

\vspace*{-.1in}
\subsection*{Creating Computer-Based Music}
\vspace*{-.05in}

Please return to the ``share'' repository for this course and type the command ``{\tt git pull}''. Now, please find the
``{\tt practical08/}'' directory and view this source code. Please note that these two Java programs will not work
correctly unless the Java compiler and virtual machine have access to the Java archive (JAR) file available in the
``{\tt lib}/'' directory. To make this file available you need to input the following type of command in your terminal
window: 

% \noindent
  {\tt export CLASSPATH=<full path>/jfugue-4.0.3.jar:.}

  To learn what to type for the ``{\tt <full path>}'' you should go into the ``{\tt lib/}'' directory in the ``{\tt
practical08/}'' directory and type the command ``{\tt pwd}''.  Then, you can place this value in the aforementioned
command. Please see the course instructor if you are having trouble with this step.

Now, please compile and run the ``{\tt FrereJacques.java}'' program. What does this program do when you run it? How does
this program use an array to create the first pattern in the song? Please note that this program will save a musical
instrument digital interface (MIDI) file in the directory where it is executed. If you would like to play the generated
file, you can type the following command in the terminal window: ``{\tt timidity frerejacues.mid}''. This file should
also play on any computer or mobile device that supports MIDI. You may listen to the musical output that this program
produces by either using the built-in speakers or connecting your own headphones.

You should notice that the ``{\tt FrereJacques.java}'' program only plays the song one time. Your task for this
practical assignment is to improve the program so that it accepts, as input from the user, the number of times that the
song should repeat. You should add this feature to the program through the inclusion of a ``{\tt for}'' loop. That is,
if the user input is stored in the variable called ``{\tt repeats}'', then the calls to the ``{\tt add}'' methods should
take place for a total of ``{\tt repeats}'' number of times. Please submit the enhanced version of this program to your
Git repository.  Students who would like to further explore music creation may also compile and run
``{\tt CrabCanon.java}''.

% To finish this assignment and earn a ``checkmark'', you should submit the
% {\tt Practical7.java} file in your Bitbucket repository by using
% the appropriate {\tt git} commands. You also need to submit an output file,
% called {\tt output}, with at least one run of your program.

% \vspace*{-.15in}
% \subsection*{General Guidelines for Practical Sessions}
% \vspace*{-.05in}
% \begin{itemize}
% \item {\bf Submit \textbf{\textit{Something}}.} Your grade for this assignment is a ``checkmark'' indicating whether you
%   did or did not complete the work and submit something to the Bitbucket repository using the ``{\tt git add}'', ``{\tt
%     git commit}'', and ``{\tt git push}'' commands.

% \item {\bf Update Your Repository Often!} You should {\tt add}, {\tt commit}, and {\tt push} your updated files each
%   time you work on them, always including descriptive messages about each code change.

% \item {\bf Review the Honor Code Policy on the Syllabus.} Remember that while you may discuss your work with other
%   students in the course, code that is nearly identical to, or merely variations on, the work of others will be
%   taken as evidence of violating the \mbox{Honor Code}.

% \end{itemize}
\end{document}
