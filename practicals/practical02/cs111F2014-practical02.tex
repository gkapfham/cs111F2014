%!TEX root=cs111F2014-practical02.tex
% mainfile: cs111F2014-practical02.tex

%!TEX root=cs111F2014-lab09.tex
% mainfile: cs111F2014-lab09.tex

% CS 111 style
% Typical usage (all UPPERCASE items are optional):
%       \input 111pre
%       \begin{document}
%       \MYTITLE{Title of document, e.g., Lab 1\\Due ...}
%       \MYHEADERS{short title}{other running head, e.g., due date}
%       \PURPOSE{Description of purpose}
%       \SUMMARY{Very short overview of assignment}
%       \DETAILS{Detailed description}
%         \SUBHEAD{if needed} ...
%         \SUBHEAD{if needed} ...
%          ...
%       \HANDIN{What to hand in and how}
%       \begin{checklist}
%       \item ...
%       \end{checklist}
% There is no need to include a "\documentstyle."
% However, there should be an "\end{document}."
%
%===========================================================
\documentclass[11pt,twoside,titlepage]{article}
%%NEED TO ADD epsf!!
\usepackage{threeparttop}
\usepackage{graphicx}
\usepackage{latexsym}
\usepackage{color}
\usepackage{listings}
\usepackage{fancyvrb}
%\usepackage{pgf,pgfarrows,pgfnodes,pgfautomata,pgfheaps,pgfshade}
\usepackage{tikz}
\usepackage[normalem]{ulem}
\tikzset{
    %Define standard arrow tip
%    >=stealth',
    %Define style for boxes
    oval/.style={
           rectangle,
           rounded corners,
           draw=black, very thick,
           text width=6.5em,
           minimum height=2em,
           text centered},
    % Define arrow style
    arr/.style={
           ->,
           thick,
           shorten <=2pt,
           shorten >=2pt,}
}
\usepackage[noend]{algorithmic}
\usepackage[noend]{algorithm}
\newcommand{\bfor}{{\bf for\ }}
\newcommand{\bthen}{{\bf then\ }}
\newcommand{\bwhile}{{\bf while\ }}
\newcommand{\btrue}{{\bf true\ }}
\newcommand{\bfalse}{{\bf false\ }}
\newcommand{\bto}{{\bf to\ }}
\newcommand{\bdo}{{\bf do\ }}
\newcommand{\bif}{{\bf if\ }}
\newcommand{\belse}{{\bf else\ }}
\newcommand{\band}{{\bf and\ }}
\newcommand{\breturn}{{\bf return\ }}
\newcommand{\mod}{{\rm mod}}
\renewcommand{\algorithmiccomment}[1]{$\rhd$ #1}
\newenvironment{checklist}{\par\noindent\hspace{-.25in}{\bf Checklist:}\renewcommand{\labelitemi}{$\Box$}%
\begin{itemize}}{\end{itemize}}
\pagestyle{threepartheadings}
\usepackage{url}
\usepackage{wrapfig}
\usepackage{hyperref}
%=========================
% One-inch margins everywhere
%=========================
\setlength{\topmargin}{0in}
\setlength{\textheight}{8.5in}
\setlength{\oddsidemargin}{0in}
\setlength{\evensidemargin}{0in}
\setlength{\textwidth}{6.5in}
%===============================
%===============================
% Macro for document title:
%===============================
\newcommand{\MYTITLE}[1]%
   {\begin{center}
     \begin{center}
     \bf
     CMPSC 111\\Introduction to Computer Science I\\
     Fall 2014\\
     %Janyl Jumadinova\\
     %\url{http://cs.allegheny.edu/~jjumadinova/111}
     \medskip
     \end{center}
     \bf
     #1
     \end{center}
}
%================================
% Macro for headings:
%================================
\newcommand{\MYHEADERS}[2]%
   {\lhead{#1}
    \rhead{#2}
    \immediate\write16{}
    \immediate\write16{DATE OF HANDOUT?}
    \read16 to \dateofhandout
    \lfoot{\sc Handed out on \dateofhandout}
    \immediate\write16{}
    \immediate\write16{HANDOUT NUMBER?}
    \read16 to\handoutnum
    \rfoot{Handout \handoutnum}
   }

%================================
% Macro for bold italic:
%================================
\newcommand{\bit}[1]{{\textit{\textbf{#1}}}}

%=========================
% Non-zero paragraph skips.
%=========================
\setlength{\parskip}{1ex}

%=========================
% Create various environments:
%=========================
\newcommand{\PURPOSE}{\par\noindent\hspace{-.25in}{\bf Purpose:\ }}
\newcommand{\SUMMARY}{\par\noindent\hspace{-.25in}{\bf Summary:\ }}
\newcommand{\DETAILS}{\par\noindent\hspace{-.25in}{\bf Details:\ }}
\newcommand{\HANDIN}{\par\noindent\hspace{-.25in}{\bf Hand in:\ }}
\newcommand{\SUBHEAD}[1]{\bigskip\par\noindent\hspace{-.1in}{\sc #1}\\}
%\newenvironment{CHECKLIST}{\begin{itemize}}{\end{itemize}}

\begin{document}
\MYTITLE{Practical 2\\11--12 September 2014\\Due in Bitbucket by midnight of the day of your practical \\ ``Checkmark'' grade}

%\MYHEADERS{Recitation 2}{Due same day, midnight}

\subsection*{Summary}

Create a Java program that prints something ``interesting'', using the {\tt git add}, {\tt git commit}, and {\tt git
  push} commands to upload it to your Git repository hosted by Bitbucket.  See the end of the assignment for a few hints
and suggestions for creating Java programs that perform output.

\subsection*{Review the Textbook}

Be sure to read section 2.1 of your book---it explains how to print some of the special characters, a topic that is also
discussed at the end of this assignment.

\subsection*{Exercise: Print Something ``Interesting''}

Create a Java program that uses a sequence of {\em no more than ten} {\tt System.out.println} statements to print
something ``interesting.'' You may {\em not} use any other features of Java, such as variables, loops, etc.  However,
you are {\em required} to use at least one of the ``escaped'' characters, such as \verb$\"$ or \verb$\\$.  Remember, it
is possible to get pictures that are taller than ten lines by using the \verb$\n$ character in your {\tt println}
statements. Please see an instructor if you have questions about this requirement.

Your program must print your name and today's date (using ``{\tt new Date()}'' in a {\tt println} statement).  This will
not count as part of your ten print statements.

% \begin{quote}
You must come up with an {\em original} design---{\em under no circumstances} should you copy a design from another
source, such as an ``ASCII Art'' web site. (However, you may look at such sites for inspiration.) Here are two examples
that you are encouraged to try.
% \end{quote}

\newpage

\subsection*{Example 1: File ``{\tt PrintName.java}''}
\begin{verbatim}
     //**********************************
     // Bob Roos
     // Recitation, 11-12 September 2014
     //
     // Prints my name
     //**********************************
     import java.util.Date;
     public class PrintName
     {
       public static void main(String[] args)
       {
          System.out.println("Bob Roos, CMPSC 111\n" + new Date() + "\n");
          System.out.println(" ____        __");
          System.out.println("  |  \\        |");
          System.out.println("  |__/   __   |__");
          System.out.println("  |  \\  /  \\  |  \\");
          System.out.println(" _|__/  \\__/ _|__/");
       }
     }
\end{verbatim}

\noindent{\bf OUTPUT:}
\begin{verbatim}
     javac PrintName.java
     java PrintName
     Bob Roos, CMPSC 111
     Wed Jan 22 21:04:41 EST 2014
     
      ____        __
       |  \        |
       |__/   __   |__
       |  \  /  \  |  \
      _|__/  \__/ _|__/
\end{verbatim}

\newpage

\subsection*{Example 2: File ``{\tt PrintFace.java}''}
\begin{verbatim}
     //**********************************
     // Janyl Jumadinova
     // Recitation 2, 22 January 2014
     //
     // Prints a face
     //**********************************
     import java.util.Date;
     public class PrintFace
     {
       public static void main(String[] args)
       {
          System.out.println("Janyl Jumadinova, CMPSC 111\n" + new Date() + "\n");
          System.out.println("   \\\\\\|||///");
          System.out.println("   /       \\");
          System.out.println("   | -- -- |");
          System.out.println("  @|  O O  |@");
          System.out.println("   |   V   |");
          System.out.println("    \\ \\_/ /");
          System.out.println("     \\___/");
       }
     } 
\end{verbatim}
\noindent{\bf OUTPUT:}
\begin{verbatim}
     javac PrintFace.java
     java PrintFace
     Janyl Jumadinova, CMPSC 111
     Wed Jan 22 21:11:55 EST 2014
     
        \\\|||///
        /       \
        | -- -- |
       @|  O O  |@
        |   V   |
         \ \_/ /
          \___/
\end{verbatim}

\newpage
\subsection*{Using Version Control Correctly}

As you are typing your program in the {\tt gvim} text editor, you should regularly save your files.  Once you have
created a preliminary version of your program, you should use the ``{\tt git add}'' command to stage it in your Git
repository.  Next, you should use the ``{\tt git commit}'' command to save it in your local repository with a version
control message.  Finally, you can run ``{\tt git push}'' to transfer your file to the Bitbucket servers.  For this
practical assignment, you do {\em not} have to hand in a hard copy of anything---just upload your Java program to
Bitbucket using the {\tt git} commands. 

Please review your ``Git Cheatsheet'' and talk with a member of the class, a course instructor, or a teaching assistant
if you do not understand how to use the the Git version control system.

\subsection*{Hints About Java Programming and Escaped Characters}

The name of your program file (for instance, ``{\tt PrintFace.java}'') must be the same as the name in the ``{\tt public
  class ...}'' statement---see earlier examples where you practiced this skill.

The characters ``\verb$\$'' (backslash) and ``\verb$"$'' (double-quote)
require special handling. To print them out, you need to put an extra
``\verb$\$'' in front of them. For instance,
\begin{quote}
The statement:\ \ \ \ \ \ \ \verb$System.out.println("backslash: \\, quote: \"");$\\
prints:\ \ \ \ \ \ \ \ \ \ \ \ \ \ \ \ \ \ 
\verb$backslash: \, quote: "$
\end{quote}

\subsection*{General Guidelines for Recitation Sessions}

\begin{itemize}

\item {\bf Experiment!} Practical sessions are for learning by doing without the pressure of grades or ``right/wrong''
  answers. So try things!  The best way to learn is by trying things out.

\item {\bf Submit \textbf{\textit{something}}.} Your grade for this assignment is a ``checkmark'' indicating whether you
  did or did not complete the work and submit something to the Bitbucket repository using the ``{\tt git add}'', ``{\tt
    git commit}'', and ``{\tt git push}'' commands.

\item {\bf Try to Finish During the Class Session.} Practical exercises are not intended to be the equal of the laboratory
  assignments. If you are simply a slow typist, I've given you until the end of the day, but ideally you should upload a
  file, even a non-working one, by the end of the class period and be finished with it.

\item
{\bf Help One Another!} 
If your neighbor is struggling and you know what
to do, offer your help. Don't ``do the work'' for them, but advise them on what
to type or how to handle things.

\item
{\bf Review the Honor Code policy on the syllabus.} Remember that you
may discuss programs with others, but programs that are nearly identical
to others will be taken as evidence of violating the Honor Code.

\end{itemize}

\end{document}
