%!TEX root=cs111F2014-practical04.tex
% mainfile: cs111F2014-practical04.tex

%!TEX root=cs111F2014-lab09.tex
% mainfile: cs111F2014-lab09.tex

% CS 111 style
% Typical usage (all UPPERCASE items are optional):
%       \input 111pre
%       \begin{document}
%       \MYTITLE{Title of document, e.g., Lab 1\\Due ...}
%       \MYHEADERS{short title}{other running head, e.g., due date}
%       \PURPOSE{Description of purpose}
%       \SUMMARY{Very short overview of assignment}
%       \DETAILS{Detailed description}
%         \SUBHEAD{if needed} ...
%         \SUBHEAD{if needed} ...
%          ...
%       \HANDIN{What to hand in and how}
%       \begin{checklist}
%       \item ...
%       \end{checklist}
% There is no need to include a "\documentstyle."
% However, there should be an "\end{document}."
%
%===========================================================
\documentclass[11pt,twoside,titlepage]{article}
%%NEED TO ADD epsf!!
\usepackage{threeparttop}
\usepackage{graphicx}
\usepackage{latexsym}
\usepackage{color}
\usepackage{listings}
\usepackage{fancyvrb}
%\usepackage{pgf,pgfarrows,pgfnodes,pgfautomata,pgfheaps,pgfshade}
\usepackage{tikz}
\usepackage[normalem]{ulem}
\tikzset{
    %Define standard arrow tip
%    >=stealth',
    %Define style for boxes
    oval/.style={
           rectangle,
           rounded corners,
           draw=black, very thick,
           text width=6.5em,
           minimum height=2em,
           text centered},
    % Define arrow style
    arr/.style={
           ->,
           thick,
           shorten <=2pt,
           shorten >=2pt,}
}
\usepackage[noend]{algorithmic}
\usepackage[noend]{algorithm}
\newcommand{\bfor}{{\bf for\ }}
\newcommand{\bthen}{{\bf then\ }}
\newcommand{\bwhile}{{\bf while\ }}
\newcommand{\btrue}{{\bf true\ }}
\newcommand{\bfalse}{{\bf false\ }}
\newcommand{\bto}{{\bf to\ }}
\newcommand{\bdo}{{\bf do\ }}
\newcommand{\bif}{{\bf if\ }}
\newcommand{\belse}{{\bf else\ }}
\newcommand{\band}{{\bf and\ }}
\newcommand{\breturn}{{\bf return\ }}
\newcommand{\mod}{{\rm mod}}
\renewcommand{\algorithmiccomment}[1]{$\rhd$ #1}
\newenvironment{checklist}{\par\noindent\hspace{-.25in}{\bf Checklist:}\renewcommand{\labelitemi}{$\Box$}%
\begin{itemize}}{\end{itemize}}
\pagestyle{threepartheadings}
\usepackage{url}
\usepackage{wrapfig}
\usepackage{hyperref}
%=========================
% One-inch margins everywhere
%=========================
\setlength{\topmargin}{0in}
\setlength{\textheight}{8.5in}
\setlength{\oddsidemargin}{0in}
\setlength{\evensidemargin}{0in}
\setlength{\textwidth}{6.5in}
%===============================
%===============================
% Macro for document title:
%===============================
\newcommand{\MYTITLE}[1]%
   {\begin{center}
     \begin{center}
     \bf
     CMPSC 111\\Introduction to Computer Science I\\
     Fall 2014\\
     %Janyl Jumadinova\\
     %\url{http://cs.allegheny.edu/~jjumadinova/111}
     \medskip
     \end{center}
     \bf
     #1
     \end{center}
}
%================================
% Macro for headings:
%================================
\newcommand{\MYHEADERS}[2]%
   {\lhead{#1}
    \rhead{#2}
    \immediate\write16{}
    \immediate\write16{DATE OF HANDOUT?}
    \read16 to \dateofhandout
    \lfoot{\sc Handed out on \dateofhandout}
    \immediate\write16{}
    \immediate\write16{HANDOUT NUMBER?}
    \read16 to\handoutnum
    \rfoot{Handout \handoutnum}
   }

%================================
% Macro for bold italic:
%================================
\newcommand{\bit}[1]{{\textit{\textbf{#1}}}}

%=========================
% Non-zero paragraph skips.
%=========================
\setlength{\parskip}{1ex}

%=========================
% Create various environments:
%=========================
\newcommand{\PURPOSE}{\par\noindent\hspace{-.25in}{\bf Purpose:\ }}
\newcommand{\SUMMARY}{\par\noindent\hspace{-.25in}{\bf Summary:\ }}
\newcommand{\DETAILS}{\par\noindent\hspace{-.25in}{\bf Details:\ }}
\newcommand{\HANDIN}{\par\noindent\hspace{-.25in}{\bf Hand in:\ }}
\newcommand{\SUBHEAD}[1]{\bigskip\par\noindent\hspace{-.1in}{\sc #1}\\}
%\newenvironment{CHECKLIST}{\begin{itemize}}{\end{itemize}}

\begin{document}
\MYTITLE{Practical 4\\2--3 October 2014\\Due in Bitbucket by midnight of the day of your practical \\ ``Checkmark'' grade}

\vspace*{-.2in}
\subsection*{Summary}
\vspace*{-.05in}

As a means to becoming more proficient when editing Java programs, you will customize your {\tt gvim} text editor,
observe and document the changes that are evident after customization, and then explore additional commands that you can
use in {\tt gvim} to quickly and effectively manipulate Java code.  Then, using the {\tt git add}, {\tt git commit}, and
{\tt git push} commands you should upload a written reflection on your experiences to your Git repository hosted by
Bitbucket.  

\vspace*{-.15in}
\subsection*{Review Additional Resources}
\vspace*{-.05in}

Since your textbook does not include a detailed discussion about {\tt gvim}, you will need to review some additional print
and online resources that explain how to become more adept at editing the text of a Java program.  First, please take
turns scanning the chapters in the ``Practical Vim'' book that the course instructor brings to the practical session. If
you want to further improve your ``essential Vim skills'' so that you can ``edit text at the speed of thought'', then
you should also visit the \url{http://vimcasts.org/} Web site to watch the screencasts and read the articles about Vim.

\vspace*{-.15in}
\subsection*{Learning About spf13-vim} 
\vspace*{-.05in}

Known as the ``ultimate Vim distribution'', spf13-vim is a configuration of the {\tt gvim} text editor designed to make it
easy for you to make your text editor highly-customized and easy-to-use.  The creator of spf13-vim, Steve Francia,
explains that this Vim distribution is ``designed for programming'', which is exactly how we use {\tt gvim} in this class. You
can learn more about spf13-vim by visiting the following Web sites.  As you are reading this material, please ask a
course instructor or a teaching assistant if you encounter something that is hard to understand.

\vspace*{-.1in}
\begin{itemize}
  \setlength{\itemsep}{.01in}
  \item The main spf13-vim Web site: \url{http://vim.spf13.com/}
  \item Explaining the benefits of spf13-vim: \url{http://spf13.com/post/why-i-use-spf13-vim/}
  \item The GitHub page for spf13-vim: \url{https://github.com/spf13/spf13-vim}
\end{itemize}

\vspace*{-.30in}
\subsection*{Installing and Using spf13-vim} 
\vspace*{-.05in}

After you have finished learning more about spf13-vim and the features that it provides, you are ready to install it
into your home account by typing the following command in your terminal.

\begin{code}
  {\tt curl https://j.mp/spf13-vim3 -L > spf13-vim.sh \&\& sh spf13-vim.sh}
\end{code}

Please make sure that you type the command exactly as it is written; if you do not input this command correctly then you
will not be able to improve your configuration of {\tt gvim}. Once you have typed this command you will see that many
programs will be downloaded and installed by a package manager.  Try to observe what is happening and take notes about
what you see. Do you recognize the names of any of the {\tt gvim} plugins as they are being installed?

Now you are ready to use your improved version of {\tt gvim}.  Using your terminal window, please go into the {\tt
practicals/practical01/} directory to find the {\tt Kinetic.java} program that you studied during the first practical
assignment. You can edit this file by typing the command ``{\tt gvim Kinetic.java}'' in your terminal. Please carefully
study the new design of {\tt gvim}---what are five ways in which it is now different from the ``default'' configuration that
you were using previously? To best answer this question, you should add
some new lines of code to {\tt Kinetic.java}.

There are many steps that you can take to further configure the {\tt gvim} text editor.  For instance, by using the
``Edit/Color Scheme'' menu item you can change the way in which {\tt gvim} uses color to highlight the syntactic
elements of a Java program. Additionally, spf13-vim installs many plugins that you can use to write Java programs more
efficiently.  If you study the main spf13-vim Web site, you will notice that some of the plugins are activated by
pressing the ``{\tt <Leader>}'' key when you are in command mode; your current configuration of {\tt gvim} uses the
comma key (i.e., ``{\tt ,}'') as the leader key. After adjusting the color scheme to suite your taste, you should learn
how to use at least one of the plugins that spf13-vim has installed.  How does your chosen plugin work?  What features
does it provide? Do you plan to use this plugin on a regular basis? Why or why not? 

Learning how to write Java programs in {\tt gvim} is similar to learning a new human language.  That is, the {\tt gvim}
text editor has its own ``language'' that you can learn.  For instance, pressing the ``{\tt =}'' key in {\tt gvim}'s
command mode will format a line of text with proper indentation.  In addition, the ``{\tt gg}'' and ``{\tt G}'' commands
respectfully move the {\tt gvim} cursor to the top and bottom of your Java program.  Knowing these facts, what do you
think that the ``{\tt gg=G}'' command does? Of course, {\tt gvim}'s language includes a wide variety of additional
commands like ``{\tt u}'', ``{\tt d}'', ``{\tt dd}'', and ``{\tt dip}''.  How do these commands allow you to manipulate
your Java program?

\vspace*{-.15in}
\subsection*{Completing the Practical Assignment}
\vspace*{-.15in}

To complete this practical assignment, you should create a {\tt practicals/practical04/} directory in your Bitbucket
repository. Then, you should use your newly configured version of {\tt gvim} to create a file called ``{\tt
  responses}''. Inside of this file, you should provide an answer to all of the questions that were posed in this
assignment sheet. That is, you can retype the question that you see in the assignment sheet and then write your answer
below it. For example, one of the questions that the assignment poses is ``how does your chosen plugin work?'' Finally,
you should turn in a screenshot that shows your re-configured {\tt gvim} editing the {\tt Kinetic.java} program.

% \newpage
\vspace*{-.15in}
\subsection*{General Guidelines for Practical Sessions}
\vspace*{-.05in}

\begin{itemize}

% \item {\bf Experiment!} Practical sessions are for learning by doing without the pressure of grades or ``right/wrong''
%   answers. So try things!  The best way to learn is by trying things out.

\item {\bf Submit \textbf{\textit{Something}}.} Your grade for this assignment is a ``checkmark'' indicating whether you
  did or did not complete the work and submit something to the Bitbucket repository using the ``{\tt git add}'', ``{\tt
    git commit}'', and ``{\tt git push}'' commands.

% \item {\bf Practice Key Laboratory Skills.} As you are completing this assignment, practice using the {\tt {\tt gvim}} text
%   editor and the Ubuntu terminal until you can easily use their most important features.  Additionally, ask
%   a teaching assistant or a course instructor to teach you some of the advanced features of {\tt {\tt gvim}} and the
%   terminal, thereby helping you to work more effectively. 

% \item {\bf Try to Finish During the Class Session.} Practical exercises are not intended to be the equal of the
%   laboratory assignments. If you are simply a slow typist, I've given you until the end of the day, but ideally you
%   should upload a file, even a non-working one, by the end of the class period. You also should ensure that, for this
%   assignment and all subsequent assignments, you can confidently upload files to your Git repository during the
%   practical session.

% \item {\bf Help One Another!} If your neighbor is struggling and you know what to do, offer your help. Don't ``do the
%   work'' for them, but advise them on what to type or how to handle things. If you are stuck on a part of this practical
%   session and you could not find any insights in either your textbook or online sources, formulate an intelligent
%   question to ask your neighbor, a teaching assistant, or a course instructor. Try to strike the right balance between
%   asking for help when you cannot solve a problem and working independently to find a solution.

\item {\bf Update Your Repository Often!} You should {\tt add}, {\tt commit}, and {\tt push} your updated files each
  time you work on them, always including descriptive messages about each code change.

\item {\bf Review the Honor Code Policy on the Syllabus.} Remember that while you may discuss your writing with other
  students in the course, text that is nearly identical to, or merely variations on, the work of others will be
  taken as evidence of violating the \mbox{Honor Code}.

\end{itemize}
\end{document}
