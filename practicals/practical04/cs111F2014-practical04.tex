%!TEX root=cs111F2014-practical04.tex
% mainfile: cs111F2014-practical04.tex

\input{111pre}
\begin{document}
\MYTITLE{Practical 4\\2--3 October 2014\\Due in Bitbucket by midnight of the day of your practical \\ ``Checkmark'' grade}

\vspace*{-.2in}
\subsection*{Summary}
\vspace*{-.05in}

As a means to becoming more proficient when editing Java programs, you will customize your {\tt gvim} text editor,
observe and document the changes that are evident after customization, and then explore additional commands that you can
use in {\tt gvim} to quickly and effectively manipulate Java code.  Then, using the {\tt git add}, {\tt git commit}, and
{\tt git push} commands you should upload a written reflection on your experiences to your Git repository hosted by
Bitbucket.  

\vspace*{-.15in}
\subsection*{Review Additional Resources}
\vspace*{-.05in}

Since your textbook does not include an informative discussion about {\tt gvim}, you will need to review some additional print
and online resources that explain how to become more adept at editing the text of a Java program.  First, please take
turns scanning the chapters in the ``Practical Vim'' book that the course instructor brings to the practical session. If
you want to further improve your ``essential Vim skills'' so that you can ``edit text at the speed of thought'', then
you should also visit the \url{http://vimcasts.org/} Web site to watch the screencasts and read the articles about Vim.

\vspace*{-.15in}
\subsection*{Learning About spf13-vim} 
\vspace*{-.05in}

Known as the ``ultimate Vim distribution'', spf13-vim is a configuration of the {\tt gvim} text editor designed to make it
easy for you to make your text editor highly-customized and easy-to-use.  The creator of spf13-vim, Steve Francia,
explains that this Vim distribution is ``designed for programming'', which is exactly how we use {\tt gvim} in this class. You
can learn more about spf13-vim by visiting the following Web sites.  As you are reading this material, please ask a
course instructor or a teaching assistant if you encounter something that is hard to understand.

\begin{itemize}
  \item The main spf13-vim Web site: \url{http://vim.spf13.com/}
  \item Explaining the benefits of spf13-vim: \url{http://spf13.com/post/why-i-use-spf13-vim/}
  \item The GitHub page for spf13-vim: \url{https://github.com/spf13/spf13-vim}
\end{itemize}

\vspace*{-.20in}
\subsection*{Installing and Using spf13-vim} 
\vspace*{-.05in}

After you have finished learning more about spf13-vim and the features that it provides, you are ready to install it
into your home account by typing the following command in your terminal.

\begin{code}
  {\tt curl https://j.mp/spf13-vim3 -L > spf13-vim.sh \&\& sh spf13-vim.sh}
\end{code}

Please make sure that you type the command exactly as it is written; if you do not input this command correctly then you
will not be able to improve your configuration of {\tt gvim}. Once you have typed this command you will see that many
programs will be downloaded and installed by a package manager.  Try to observe what is happening and take notes about
what you see. Do you recognize the names of any of the {\tt gvim} plugins as they are being installed?



\vspace*{-.15in}
\subsection*{General Guidelines for Practical Sessions}
\vspace*{-.05in}
\begin{itemize}

\item {\bf Experiment!} Practical sessions are for learning by doing without the pressure of grades or ``right/wrong''
  answers. So try things!  The best way to learn is by trying things out.

\item {\bf Submit \textbf{\textit{Something}}.} Your grade for this assignment is a ``checkmark'' indicating whether you
  did or did not complete the work and submit something to the Bitbucket repository using the ``{\tt git add}'', ``{\tt
    git commit}'', and ``{\tt git push}'' commands.

\item {\bf Practice Key Laboratory Skills.} As you are completing this assignment, practice using the {\tt {\tt gvim}} text
  editor and the Ubuntu terminal until you can easily use their most important features.  Additionally, ask
  a teaching assistant or a course instructor to teach you some of the advanced features of {\tt {\tt gvim}} and the
  terminal, thereby helping you to work more effectively. 

\item {\bf Try to Finish During the Class Session.} Practical exercises are not intended to be the equal of the
  laboratory assignments. If you are simply a slow typist, I've given you until the end of the day, but ideally you
  should upload a file, even a non-working one, by the end of the class period. You also should ensure that, for this
  assignment and all subsequent assignments, you can confidently upload files to your Git repository during the
  practical session.

\item {\bf Help One Another!} If your neighbor is struggling and you know what to do, offer your help. Don't ``do the
  work'' for them, but advise them on what to type or how to handle things. If you are stuck on a part of this practical
  session and you could not find any insights in either your textbook or online sources, formulate an intelligent
  question to ask your neighbor, a teaching assistant, or a course instructor. Try to strike the right balance between
  asking for help when you cannot solve a problem and working independently to find a solution.

\item {\bf Update Your Repository Often!} You should {\tt add}, {\tt commit}, and {\tt push} your updated files each
  time you work on them, always including descriptive messages about each code change.

\item {\bf Review the Honor Code Policy on the Syllabus.} Remember that while you may discuss programs with other
  students in the course, programs that are nearly identical to, or merely variations on, the work of others will be
  taken as evidence of violating the Honor Code.
\end{itemize}
\end{document}
