%!TEX root=cs111F2014-practical04.tex
% mainfile: cs111F2014-practical04.tex

%!TEX root=cs111F2014-lab09.tex
% mainfile: cs111F2014-lab09.tex

% CS 111 style
% Typical usage (all UPPERCASE items are optional):
%       \input 111pre
%       \begin{document}
%       \MYTITLE{Title of document, e.g., Lab 1\\Due ...}
%       \MYHEADERS{short title}{other running head, e.g., due date}
%       \PURPOSE{Description of purpose}
%       \SUMMARY{Very short overview of assignment}
%       \DETAILS{Detailed description}
%         \SUBHEAD{if needed} ...
%         \SUBHEAD{if needed} ...
%          ...
%       \HANDIN{What to hand in and how}
%       \begin{checklist}
%       \item ...
%       \end{checklist}
% There is no need to include a "\documentstyle."
% However, there should be an "\end{document}."
%
%===========================================================
\documentclass[11pt,twoside,titlepage]{article}
%%NEED TO ADD epsf!!
\usepackage{threeparttop}
\usepackage{graphicx}
\usepackage{latexsym}
\usepackage{color}
\usepackage{listings}
\usepackage{fancyvrb}
%\usepackage{pgf,pgfarrows,pgfnodes,pgfautomata,pgfheaps,pgfshade}
\usepackage{tikz}
\usepackage[normalem]{ulem}
\tikzset{
    %Define standard arrow tip
%    >=stealth',
    %Define style for boxes
    oval/.style={
           rectangle,
           rounded corners,
           draw=black, very thick,
           text width=6.5em,
           minimum height=2em,
           text centered},
    % Define arrow style
    arr/.style={
           ->,
           thick,
           shorten <=2pt,
           shorten >=2pt,}
}
\usepackage[noend]{algorithmic}
\usepackage[noend]{algorithm}
\newcommand{\bfor}{{\bf for\ }}
\newcommand{\bthen}{{\bf then\ }}
\newcommand{\bwhile}{{\bf while\ }}
\newcommand{\btrue}{{\bf true\ }}
\newcommand{\bfalse}{{\bf false\ }}
\newcommand{\bto}{{\bf to\ }}
\newcommand{\bdo}{{\bf do\ }}
\newcommand{\bif}{{\bf if\ }}
\newcommand{\belse}{{\bf else\ }}
\newcommand{\band}{{\bf and\ }}
\newcommand{\breturn}{{\bf return\ }}
\newcommand{\mod}{{\rm mod}}
\renewcommand{\algorithmiccomment}[1]{$\rhd$ #1}
\newenvironment{checklist}{\par\noindent\hspace{-.25in}{\bf Checklist:}\renewcommand{\labelitemi}{$\Box$}%
\begin{itemize}}{\end{itemize}}
\pagestyle{threepartheadings}
\usepackage{url}
\usepackage{wrapfig}
\usepackage{hyperref}
%=========================
% One-inch margins everywhere
%=========================
\setlength{\topmargin}{0in}
\setlength{\textheight}{8.5in}
\setlength{\oddsidemargin}{0in}
\setlength{\evensidemargin}{0in}
\setlength{\textwidth}{6.5in}
%===============================
%===============================
% Macro for document title:
%===============================
\newcommand{\MYTITLE}[1]%
   {\begin{center}
     \begin{center}
     \bf
     CMPSC 111\\Introduction to Computer Science I\\
     Fall 2014\\
     %Janyl Jumadinova\\
     %\url{http://cs.allegheny.edu/~jjumadinova/111}
     \medskip
     \end{center}
     \bf
     #1
     \end{center}
}
%================================
% Macro for headings:
%================================
\newcommand{\MYHEADERS}[2]%
   {\lhead{#1}
    \rhead{#2}
    \immediate\write16{}
    \immediate\write16{DATE OF HANDOUT?}
    \read16 to \dateofhandout
    \lfoot{\sc Handed out on \dateofhandout}
    \immediate\write16{}
    \immediate\write16{HANDOUT NUMBER?}
    \read16 to\handoutnum
    \rfoot{Handout \handoutnum}
   }

%================================
% Macro for bold italic:
%================================
\newcommand{\bit}[1]{{\textit{\textbf{#1}}}}

%=========================
% Non-zero paragraph skips.
%=========================
\setlength{\parskip}{1ex}

%=========================
% Create various environments:
%=========================
\newcommand{\PURPOSE}{\par\noindent\hspace{-.25in}{\bf Purpose:\ }}
\newcommand{\SUMMARY}{\par\noindent\hspace{-.25in}{\bf Summary:\ }}
\newcommand{\DETAILS}{\par\noindent\hspace{-.25in}{\bf Details:\ }}
\newcommand{\HANDIN}{\par\noindent\hspace{-.25in}{\bf Hand in:\ }}
\newcommand{\SUBHEAD}[1]{\bigskip\par\noindent\hspace{-.1in}{\sc #1}\\}
%\newenvironment{CHECKLIST}{\begin{itemize}}{\end{itemize}}

\begin{document}
\MYTITLE{Practical 4\\2--3 October 2014\\Due in Bitbucket by midnight of the day of your practical \\ ``Checkmark'' grade}

\vspace*{-.2in}
\subsection*{Summary}
\vspace*{-.05in}

As a means for becoming more proficient when editing Java programs, you will customize your GVim text editor, observe
and document the changes that are evident after customization, and then explore additional commands that you can use in
GVim to quickly manipulate Java code.  Then, using the {\tt git add}, {\tt git commit}, and {\tt git push} commands you
should upload a written reflection on your experiences to your Git repository hosted by Bitbucket.  

\vspace*{-.15in}
\subsection*{Review Additional Resources}
\vspace*{-.05in}

Since your textbook does not include an informative discussion about GVim, you will need to review some additional print
and online resources that explain how to become more adept at editing the text of a Java program.  First, please take
turns scanning the chapters in the ``Practical Vim'' book that the course instructor brings to the practical session. If
you want to further improve your ``essential Vim skills'' so that you can ``edit text at the speed of thought'', then
you should also visit the \url{http://vimcasts.org/} Web site to watch the screencasts and read the articles about Vim.

\vspace*{-.15in}
\subsection*{Exercise: Implement a ``Mad Libs'' Program} 
\vspace*{-.05in}
Your program should ask the user to enter words and numbers, then print out a story using those words and numbers. The
numbers should be used to calculate something which will also be printed out. See Figure \ref{mad} for an example---but,
of course, you should create your own story!

\begin{figure}[tb]
\begin{Verbatim}[commandchars=\\\{\}]
aldenv27:gkapfham$ \textcolor{red}{javac MadLib.java}
aldenv27:gkapfham$ \textcolor{red}{java MadLib}
Gregory M. Kapfhammer, Practical 3
Thu Sep 17 13:11:17 EST 2014

Enter a singular noun: \textcolor{red}{noggin}
Enter an adjective: \textcolor{red}{verboten}
Enter another adjective: \textcolor{red}{glitzy}
Enter a non-zero whole number: \textcolor{red}{32}
Enter another non-zero whole number: \textcolor{red}{42}
Enter any number: \textcolor{red}{5.43}
Enter a singular verb: \textcolor{red}{snooze}

----------------------------------

Third-Grade Word Problem

If you own 32 verboten noggins,
and you purchase 42 glitzy noggins,
how many more noggins do you need to snooze 5.43 noggins?

Answer: You need -68.57 more noggins.
\end{Verbatim}
\vspace*{-.1in}
\caption{Sample ``Mad Libs'' output with user input highlighted in red.}
\label{mad}
\end{figure}

% \begin{sloppypar}
\noindent Remember to ``{\tt import java.util.Scanner}'' at the top of your Java program and to declare a {\tt Scanner}
variable (named, for instance, {\tt scan}---but you can name it something else if you want). Please use statements like
``{\tt... = scan.next()}'', ``{\tt... = scan.nextInt()}'', and ``{\tt... = scan.\\nextDouble()}'' to read in strings,
integers, and double values. Your program should produce output that is neat and your program's source code must make
good use of white space and labeling.
% \end{sloppypar}

\vspace*{-.15in}
\subsection*{General Guidelines for Practical Sessions}
\vspace*{-.05in}
\begin{itemize}

\item {\bf Experiment!} Practical sessions are for learning by doing without the pressure of grades or ``right/wrong''
  answers. So try things!  The best way to learn is by trying things out.

\item {\bf Submit \textbf{\textit{Something}}.} Your grade for this assignment is a ``checkmark'' indicating whether you
  did or did not complete the work and submit something to the Bitbucket repository using the ``{\tt git add}'', ``{\tt
    git commit}'', and ``{\tt git push}'' commands.

\item {\bf Practice Key Laboratory Skills.} As you are completing this assignment, practice using the {\tt gvim} text
  editor and the Ubuntu terminal until you can easily use their most important features.  Additionally, ask
  a teaching assistant or a course instructor to teach you some of the advanced features of {\tt gvim} and the
  terminal, thereby helping you to work more effectively. 

\item {\bf Try to Finish During the Class Session.} Practical exercises are not intended to be the equal of the
  laboratory assignments. If you are simply a slow typist, I've given you until the end of the day, but ideally you
  should upload a file, even a non-working one, by the end of the class period. You also should ensure that, for this
  assignment and all subsequent assignments, you can confidently upload files to your Git repository during the
  practical session.

\item {\bf Help One Another!} If your neighbor is struggling and you know what to do, offer your help. Don't ``do the
  work'' for them, but advise them on what to type or how to handle things. If you are stuck on a part of this practical
  session and you could not find any insights in either your textbook or online sources, formulate an intelligent
  question to ask your neighbor, a teaching assistant, or a course instructor. Try to strike the right balance between
  asking for help when you cannot solve a problem and working independently to find a solution.

\item {\bf Update Your Repository Often!} You should {\tt add}, {\tt commit}, and {\tt push} your updated files each
  time you work on them, always including descriptive messages about each code change.

\item {\bf Review the Honor Code Policy on the Syllabus.} Remember that while you may discuss programs with other
  students in the course, programs that are nearly identical to, or merely variations on, the work of others will be
  taken as evidence of violating the Honor Code.
\end{itemize}
\end{document}
